\documentclass[11pt, authoryear]{elsarticle}

\usepackage[hyphens]{url}                
\usepackage{hyperref}
\usepackage[hyphenbreaks]{breakurl}
\usepackage{rotating}
\usepackage{wrapfig}
\usepackage{pdflscape}
\usepackage{fixltx2e}
\usepackage{graphicx}
\usepackage{amsmath}
\usepackage{amsfonts}
\usepackage[section]{placeins}
\usepackage{dirtree}
\usepackage{siunitx}
\usepackage{afterpage}
\usepackage{pdflscape}
\usepackage{svg}
\usepackage[export]{adjustbox}


\usepackage{booktabs}
\usepackage{dcolumn}
\makeatletter
\newcolumntype{D}[3]{>{\textfont0=\the\font\DC@{#1}{#2}{#3}}c<{\DC@end}}
\makeatother

\newcolumntype{L}{>{$}l<{$}}

% \usepackage{bibentry}

\sisetup{detect-all}

\sloppy
\usepackage{scalerel,stackengine}

\usepackage{caption}
\usepackage[draft]{todonotes}

\captionsetup{skip=0pt}
\usepackage[utf8]{inputenc}

\usepackage[T1]{fontenc}
\usepackage{csquotes}




\usepackage{floatrow}

\usepackage{listings}
\usepackage{xcolor}
\usepackage{colortbl}

\usepackage{crimson}
\usepackage{microtype}


\usepackage{fancyhdr}
\usepackage{setspace}
\singlespace
\usepackage{longtable}
\usepackage{subfig}
\usepackage[a4paper, total={18cm, 24cm}]{geometry}

\pagestyle{fancy}
\fancyhf{}
\renewcommand{\headrulewidth}{0pt}
\renewcommand{\maketitle}{}

\usepackage{enumitem}
\setlist[itemize]{topsep=0pt,itemsep=0pt,parsep=0pt,partopsep=0pt}

\usepackage{multicol}
\setlength\multicolsep{0pt}

\usepackage{array}
\usepackage{caption}
\usepackage{graphicx}
\usepackage{siunitx}
\usepackage[normalem]{ulem}
\usepackage{colortbl}
\usepackage{multirow}
\usepackage{hhline}
\usepackage{calc}
\usepackage{tabularx}
\usepackage{threeparttable}
\usepackage{wrapfig}
\usepackage{adjustbox}
\usepackage{hyperref}


\newlist{propertyList}{itemize}{1}
\setlist[propertyList]{
  label=\textbullet,
  noitemsep,
  leftmargin=10pt,
  before=\begin{multicols}{3},
  after=\end{multicols}
  }

\rfoot {\thepage}

\listfiles

\begin{filecontents}{refs_poetics_sup.bib}
@misc{Blanchet_2017_conversions,
  title = {Prices and currency conversions in WID.world},
  author = {Blanchet, Thomas},
  url = {https://wid.world/document/convert-wid-world-series/},
  year = {2017},
}

@article{Luedecke_etal_2021_performance,
  author = {Lüdecke, Daniel and Ben-Shachar, Mattan and Patil, Indrajeet and Waggoner, Philip and Makowski, Dominique},
  title = {performance: An R Package for Assessment, Comparison and Testing of Statistical Models},
  journal = {Journal of Open Source Software},
  year = {2021},
  volume = {6},
  number = {60},
  month = {Apr},
  pages = {3139},
  issn = {2475-9066},
  doi = {10.21105/joss.03139},
  url = {http://dx.doi.org/10.21105/joss.03139},
  publisher = {The Open Journal},
}

@misc{WID_2021_WID,
  title = {World Inequality Database},
  url = {https://wid.world/},
  year = {2021},
  author = {{World Inequality Database}},
}

\end{filecontents}


\setlength{\parindent}{1.2cm}
\author{ }
\date{\today}
\title{}
\hypersetup{
 pdfauthor={ },
 pdftitle={},
 pdfkeywords={},
 pdfsubject={},
 pdfcreator={Emacs 29.1 (Org mode 9.6.7)}, 
 pdflang={English}}
\begin{document}



\section*{Supplementary online materials}



\appendix
\setcounter{page}{1}


\counterwithin{table}{subsection}
\counterwithin{figure}{subsection}
\renewcommand{\thesubsection}{\Alph{subsection}}



\subsection{Data Coverage}
\label{app_data_coverage}
\begin{figure}[htbp]
\centering
\includegraphics[width=18cm]{plt_v27_vrbl_cycnt.pdf}
\caption{\label{fig:vrbl_cycnt}Number of countries with per year per variable}
\end{figure}



\begin{figure}[htbp]
\centering
\includegraphics[width=18cm]{plt_v27_cbn_cycnt.pdf}
\caption{\label{fig:cbn_cycnt}Number of countries per year per variable combination}
\end{figure}



Figure \ref{fig:vrbl_cycnt} shows the country-year coverage of the main variables (other HNWI thresholds and inequality shares follow the ones depicted).
In particular it can be seen that the coverage of wealth variables in the WID improves substantially from 1995 onwards, a pattern that is to a lesser extent also visible in indicators of cultural spending and top marginal income tax rates.
Given this state of data coverage and the fact that five years of subsequent data are required for lag length optimization, the observation period for the largest proportion of countries starts in the year 2000 or later  (figure \ref{fig:cbn_cycnt}, the start 1995 is set by the availability of Artnews collector ranking data from 1990 onwards).





\subsection{Data processing}
\label{app_data_processing}
\subsubsection{Cultural spending data source combination}


This combination of multiple data sources requires the harmonization of different reporting standards: 
Whereas the the IMF and Eurostat report data exclusively as "Total government expenditure" (TLYCG), the UN uses "Final consumption expenditure" (P3CG); the OECD reports data in both formats ("Total government expenditure" is calculated from "Final consumption expenditure" as well as a number of other items, such as compensation of employees and subsidies). 
Moreoever, within each format minor variations exist between data sources, the data for a country-year is thus chosen in order of OECD followed by UN for P3CG, and OECD followed by IMF followed by Eurostat for TLYCG.
As the overall goal is to create a complete picture of government expenditure, Total government expenditure is estimated from Final consumption expenditure for country-years where data is only available for the latter.
For countries where TLYCG and P3CG series have some overlap and years exist with P3CG data but not TLYCG data, a country-specific scaler to convert P3CG to TLYCG is constructed from overlapping years, which is then used to impute TLYCG for the years in which only P3CG data is available.
For countries with only P3CG data, the average ratio of all country-years with both P3CG and TLYCG data is chosen to impute TLYCG (as on average P3CG is 58\% of TLYCG, the average scaler is 1/0.58 = 1.72). 
Amounts are reported in current local currency units and converted to 2021 USD using price indices and purchasing power parity adjusted exchange rates from the World Inequality Database (WID,  \citeyear{WID_2021_WID}).

\subsubsection{Imputation}


Due to the exploratory approach of testing variables at lag lengths varying from one to five years, missing values can potentially substantially limit the number of country years as a single missing value leads to the exclusion of the next five years.
To avoid such loss of data, missing values in the country year time series which are parts of gaps of up to three years are linearly imputed.
This primarily concerns government cultural spending (25 country years imputed), and to a lesser extent wealth inequality measures, HNWI measures and population size (7, 4 and 3 country years imputed, respectively).


Furthermore, it was not possible to find the exact closing years for 25 private museums which were found to be no longer open.
These cases constitute a challenge for calcuating accurate density measures: 
Leaving out these museums entirely would lead to underestimated density estimates, while treating these museums as remaining open would overestimate private museum density as they were observed to be no longer open. 
Either method can substantially bias density estimates as in particular in countries with only a few private museums, a private museum more or less can have large impacts on per capita private museum rates.
To be able to still use these cases in density estimates, closing year was imputed via linear regression based on the relationship between number of years opened and closing year of the museums for which both were available (n=53, R\textsuperscript{2} = 0.68).
While imputed closing years are likely not always accurate, the resulting density estimates are likely more accurate than they would have been if closed museums had been excluded completely or treated as having remained open, especially given the high R\textsuperscript{2} of imputation regression used for the traning data.


\subsubsection{Data restrictions}

A number of observations have been excluded for various reasons:
The exchange rates for Zimbabwe and Venezuela were deemed unrealistic (Venezuela is discussed by \citet{Blanchet_2017_conversions}); both countries were therefore excluded.
The HNWI indicators for the United Arab Emirates, Saudi Arabia and Qatar display a strong decline between 2000 and 2010 of which both the starting value and the strength of the decline are extremely high (starting at several standard deviations above the mean and declining to average values) and hence likely caused by data issues; the countries are therefore excluded. 
Furthermore, a number of country-years have been removed for Yemen due to negative cultural spending.
For a number of years the wealth gini coefficient of South Africa was larger than 1; it has therefore been set to a ceiling of 0.95.
Finally, Iceland, the Bahamas, Monaco and Liechtenstein have been excluded as these countries' small population results in an extremely high rate of Artnews top 200 collectors per capita (and in the case of Iceland, also an extremely high rate of modern/contemporary art museums in 1990).
Since the number of Artnews collectors is a discrete count variable it is unable to provide accurate measures in countries with very small populations which justifies the exclusion of these countries on methodological grounds.







\subsection{Control Variables}
\label{app_controls}



The coefficient of GDP per capita is generally positive. 
In the two larger datasets, it is significant and also substantially larger with values of 0.79 and 0.88 for "DS --CuSp" and "DS --CuSP/TMITR" respectively, compared to 0.14 for "DS all IVs".
This between-dataset variation in coefficient size could indicate mediation of GDP through cultural spending (cf. appendix \ref{app_mediation} and figure \ref{fig:oucoefchng}); however further investigation would be required to more clearly determine the mechanism at play. 
Economic growth is also positive and significant in all data sets.





In all datasets on the country and global level, measures of density (legitimacy) and density squared (competition) correspond to the density-dependence paradigm, i.e. an inverted U-shaped relationship between density and private museum founding (indicated by the significant negative coefficient of the squared density measure on the country and global level for all datasets). 
Private museum founding thus is generally compatible with the density dependence paradigm, according to which potential founders are motivated by their peers' decision to establish a private museum up to the point where private museum numbers increased to the extent that additional founding does not appear advantageous anymore.
However, density in neighboring countries does not fit this trend.
While the coefficients are insignificant in the two larger datasets, in DS all IVs it describes a U-shaped relationship, corresponding to higher rates of founding at the extreme ends of the scale.
Given the right-skewed distribution of neighbor density increasing density in neighboring countries primarily decreases private museum founding in the focal country.
Furthermore, while prospective founders might ultimately be disincentivized by growing private museum numbers, there is no evidence for a demotivating effect of museum closures, as the coefficient of the cumulative number of museum closures is not significant.










\subsection{Coefficient Distribution}
\label{app_coefdrbn}
\begin{figure}[htbp]
\centering
\includegraphics[width=18cm]{plt_v27_coef_violin.pdf}
\caption{\label{fig:coef_violin}Distribution of coefficient point estimates (Gaussian kernel density estimate; bandwidth = 0.04)}
\end{figure}

\FloatBarrier


\subsection{Model improvement given inclusion of variables}
\label{app_llrt}
\begin{figure}[htbp]
\centering
\includegraphics[width=18cm]{plt_v27_oneout_llrt_lldiff.pdf}
\caption{\label{fig:oneout_llrt_lldiff}Model improvement given variable inclusion (Gaussian kernel density estimate; bandwidth = 0.4)}
\end{figure}


\begin{figure}[htbp]
\centering
\includegraphics[width=18cm]{plt_v27_oneout_llrt_z.pdf}
\caption{\label{fig:oneout_llrt_z}Distribution of Z-score of log-likelihood ratio test p-value (Gaussian kernel density estimate; bandwidth = 0.1)}
\end{figure}

To investigate whether a variable improves the model, a comparison is made between the full model and the full model without the variable in question.
For each dataset there are 36 models (due to variables choices for HWNI (4 different thresholds) and inequality measures (1 of 3 for both wealth and income inequality)), resulting in 108 models in total. 
For each variable in each of these models a reduced model is constructed by removing the variable in question and comparing model fit to the full model.
If a variable has a squared term or interaction, it is removed together with the main term.
Furthermore additional reduced models is constructed, namely one without the four density variables (country and global density linear and squared), as well as one without the density variables and closings.
Given that the datasets differ in their number of variables, a different number of reduced models is calculated per dataset, in particular 684 for "DS all IVs", 612 for "DS --CuSp", and 540 for "DS --CuSp/TMITR". 
The lags of the reduced models are not optimized due to computational limitations. 



Figure \ref{fig:oneout_llrt_lldiff} shows the distribution of differences in log-likelihood between the full and reduced models per variable and dataset.
Furthermore, a likelihood ratio test (\(LR = 2[LL_{reduced} - LL_{full}]\)) is conducted to compare each reduced to its corresponding full model.
The likelihood ratio statistic follows a Chi-square distribution; its corresponding p-value was converted to a z-score to facilitate interpretation.
The distribution of z-scores per variable and dataset is shown in figure \ref{fig:oneout_llrt_z}.
Both analysis correspond in large parts to the results of the main regression analysis insofar as variables with significant coefficient correspond to significant and/or substantial model improvements.
There are however a few exceptions, such as tax deductibility of donations in "DS -CuSp/TMITR", GDP per capita in "DS all IVs" as well as some inequality variables in "DS all IVs" in which a significant coefficient does not always correspond to a significant model improvement. 

\FloatBarrier



\subsection{Influence of variable inclusion on regression coefficients}
\label{app_mediation}
I furthermore analyze the coefficients of the restricted models to investigate potential mediation; results are presented in figure \ref{fig:oucoefchng}.
The variables (or variable sets of all density variables and all density variables plus closures) that are added are placed on the x-axis, the coefficients of the full model are placed on the y-axis;
Each point shows the average difference between the coefficient of the full and the restricted model and can be understood as the effect that adding variable v\textsubscript{x} to the model has on the coefficient of variable v\textsubscript{y}.
For example, if GDP has a coefficient of 0.5 in the full model and one of 0.7 in a restricted model (e.g. one without cultural spending), the difference is 0.5 - 0.7 = -0.2; thus adding cultural spending to the model results in an decrease of the GDP coefficient by 0.2. 
Positive coefficient changes (i.e. a larger coefficients in the full model than in the restricted model) are colored as red, negative coefficient changes as blue; points are furthermore scaled by the absolute coefficient size to compare both positive and negative changes. 


A number of findings can be gleaned from this analysis.
Firstly, wealth and income inequality appear "mutually reinforcing".
The inclusion of income inequality increases the coefficients of wealth inequality (which is positive in the full model) and the inclusion of wealth inequality decreases further the negative coefficient of income inequality (which in the full model is negative).
This unexpected pattern (as well as the divergent inequality in general) clearly calls for further research to disentangle relations of inequality.


Secondly, a number of variables appear to partly mediate GDP.
The coefficient of GDP decreases as other variables are added, which indicates that part of the effect is mediated through these variables.
This in particular concerns the effects of density, tax incentives and cultural spending, and to a lesser extent the effect of inequalities (for "DS --CuSp" and "DS --CuSp/TMITR") and some HNWI measures (for "DS all IVs").
Conversly, adding GDP to a model in which it was not included before reduces the coefficients of HNWIs and museums of modern/contemporary art.


Finally, there are a number of substantial coefficient changes given inclusion which do not offer a straightforward explanation, such as decreases of the coefficient of modern and contemporary art museums when including income inequality measures, increases of the linear density term when adding wealth inequality variables, increases of income inequality coefficients when adding density measures, as well as changes to the intercept in both directions when adding different variables. 
Exploring the various cases of potential mediation (or other forms of variable relations) is beyond the scope of the paper, but constitutes a promising start for future research.



\subsubsection{Insignifiance of HNWI effects}


Next to the substantial argument for the insignifiance of elite wealth effects proposed in the main text, namely a theoretical expectation based on selection bias, a more technical possibility is that HNWI effects might be controlled away by other variables that also measure HWNI rates, such as GDP per capita, Artnews collector rates, wealth inequality or private museum density.
The evidence for this is strongest in the case of GDP, as its addition shrinks HNWI coefficients in the two larger datasets (figure \ref{fig:oucoefchng}).
However since in the full model the HNWI coefficients are negative the addition of GDP does not reduce the strength of the association from substantial to negative, but rather from unsubstantial to negative and (marginally) significant. 
Furthermore, little evidence of HNWI coefficient change exists following the addition of wealth inequality and Artnews collectors, and even less for private museum density the addition of which actually marginally increases the HNWI coefficients.
As the models which form the basis for figure \ref{fig:oucoefchng} exclude only one variable at a time, it cannot be ruled out completely that a different specification without any variables that might also capture the elite wealth mechanism might result in substantial main HWNI effects.
However, the overall relative small size of coefficient changes by single variable exclusion does not provide strong evidence for overcontrolling as the explanation of insignificant HNWI effects. 


\begin{landscape}

\begin{figure}[htbp]
\centering
\includegraphics[width=24cm]{plt_v27_oucoefchng.pdf}
\caption{\label{fig:oucoefchng}Coefficient changes given addition of other variables}
\end{figure}

\end{landscape}




\subsection{Lag Choice}
\label{app_lagchoice}
\begin{figure}[htbp]
\centering
\includegraphics[width=18cm]{plt_v27_lag_dens.pdf}
\caption{\label{fig:lag_dens}Lag choice distribution}
\end{figure}

Figure \ref{fig:lag_dens} shows the distribution of the lag of the coefficient after optimization.
As often time lags different from one year are chosen (which would likely constitute the default if they were not varied), it can be seen that allowing the lag to vary substantially increases model fit. 
It furthermore seems to be the case that the HNWI coefficients (which are not significant) vary the most in regards to their lag choice (which is plausible since a non-substantial overall effect could imply that the particular lag does not matter much). 




\subsection{Multicollinearity}
\label{app_vif}
\begin{figure}[htbp]
\centering
\includegraphics[width=18cm]{plt_v27_vif.pdf}
\caption{\label{fig:vif}Distribution of VIF estimates (Gaussian kernel density estimate; bandwidth = 0.1)}
\end{figure}


VIFs were calculated for the best-fitting model of each variable set and dataset (108 models in total given 1 of 4 HWNI variables \texttimes{} 1 of 3 income inequality variables \texttimes{} 1 of 3 wealth inequality variables \texttimes{} 1 of 3 datasets) using the R \texttt{performance} package \citep{Luedecke_etal_2021_performance}. 
As squared variables and interactions can result in high VIFs without substantial collinearity, I calculate VIFs once for the full model and once after excluding squared variables and interactions.
Figure \ref{fig:vif} shows the distribution of the variance inflation factors.
While VIFs can be substantial when including squared variables and interactions, no multicollinearity issues are present when focusing only on the linear variables (all VIFs < 10, all VIFs except global density (which after removing squared variables is still based on the same data as global density) < 5).



\subsection{Longitudinal development}
\label{app_velp}
The within-country changes were analyzed to characterize the development of the longitudinal variables over the observation period.
In particular, for each variable a separate regression model was run which regresses the variable in question at lag 0 against year while allowing slopes and intercepts to vary by country (year is set to 0 in 1995, the beginning of the observation period).
Results are presented in figure \ref{fig:velp}.
The histogram shows the distribution of country slopes, while the dot and whiskper shows the overall slope estimate with a 95\% confidence interval.
For example the overall slope of top marginal income tax rates is -0.009 (indicating an average yearly decrease of top marginal income tax rates by 0.13 percentage points), however the histogram shows that countries can substantially diverge from this overall slope: 
Slopes of countries (with at least 20 years of data) can range from a minimum of -0.116 (a yearly decrease by 1.6 percentage points in the case of Hungary) to a maximum of 0.063 (a yearly increase of 0.9 percentage points in the case of Portugal), with 25\% and 75\% slope quantiles corresponding to -0.024 (-0.34\%) and 0.007 (0.10\%), respectively.
Furthermore, the correlation between slope and intercept is included:
A positive correlation indicates that countries with higher constants (i.e. predicted value in 1995) achieve higher growth over the observation period than countries with lower constants (thereby "extending their lead"), whereas negative correlations indicate that countries with lower constants experience higher growth (thereby "catching up").


A number of findings can be gleaned from this analysis.
First, country slopes can diverge substantially from the overall mean development on all variables, which should serve as a reminder to be cautious of treating global developments as homogenous between countries. 
This also further supports the methodological choice of using country-year as the unit of analysis instead of treating private museums as a homogenous global development; which could be further refined in the future by integrating random slopes into the main regression analysis.
Secondly it can be seen that top marginal income tax rates have on average been declining, while cultural spending as actually increased in most countries, which questions the characterization of government expenditure reduction but only to some degree as these characterizations focus primarily on countries with higher cultural spending, where due to the negative correlation between intercept and slope cultural spending has indeed decreased.
Thus the value of using a large number of countries rather than case studies which allows to characterize wider developments becomes especially clear.
Third, inequality measures as well show a wide range of developments.
While some average non-zero slopes are found (such as a an overall reduction of income inequality in DS -CuSp/TMITR and and increase in wealth inequality in DS all IVs), the variation even in these cases is still considerable. 
Fourth, HNWI indicators develop almost never negatively, although most countries see comparatively little growth (this is not a measurement artifact of assigning zero when no quantile is above the HNWI threshold in question as it occurs also at the lowest of 1M, where most country-years have non-zero values).
Furthermore average increases appear to be driven by a comparatively small number of countries with relatively large increases. 
Finally, GPD and private museum density see similar patterns of almost no decline with growth being concentrated in a handful of countries, while art collector rates are highly centered around zero, indicating stability (at zero art collectors) for most countries. 



\begin{landscape}

\begin{figure}[htbp]
\centering
\includegraphics[width=24cm]{plt_v27_velp.pdf}
\caption{\label{fig:velp}Results of regressing longitudinal variables on year}
\end{figure}

\end{landscape}


% \begin{sloppypar}
% \printbibliography[segment=\therefsegment]
% \end{sloppypar}

\bibliographystyle{plainnat}
\bibliography{refs_poetics_sup.bib}

\end{document}