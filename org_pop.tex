% Created 2023-07-11 di 10:51
% Intended LaTeX compiler: pdflatex
\documentclass[11pt]{article}

\usepackage[hyphens]{url}                
\usepackage{hyperref}
\usepackage[hyphenbreaks]{breakurl}
\usepackage{rotating}
\usepackage{wrapfig}
\usepackage{pdflscape}
\usepackage{fixltx2e}
\usepackage{graphicx}
\usepackage{amsmath}
\usepackage{amsfonts}
\usepackage[section]{placeins}
\usepackage{dirtree}
\usepackage{siunitx}
\usepackage{afterpage}
\usepackage{pdflscape}
\usepackage{svg}
\usepackage[export]{adjustbox}


\usepackage{booktabs}
\usepackage{dcolumn}
\makeatletter
\newcolumntype{D}[3]{>{\textfont0=\the\font\DC@{#1}{#2}{#3}}c<{\DC@end}}
\makeatother


\newcolumntype{L}{>{$}l<{$}}

\usepackage{bibentry}

\sisetup{detect-all}

\sloppy
\usepackage{scalerel,stackengine}

\stackMath
\newcommand\reallywidehat[1]{%
\savestack{\tmpbox}{\stretchto{%
  \scaleto{%
    \scalerel*[\widthof{\ensuremath{#1}}]{\kern-.6pt\bigwedge\kern-.6pt}%
    {\rule[-\textheight/2]{1ex}{\textheight}}%WIDTH-LIMITED BIG WEDGE
  }{\textheight}% 
}{0.5ex}}%
\stackon[1pt]{#1}{\tmpbox}%
}

\usepackage{caption}
\usepackage[draft]{todonotes}

\captionsetup{skip=0pt}
\usepackage[utf8]{inputenc}
\usepackage[style=apa, backend=biber]{biblatex} 
\usepackage[english, american]{babel}
\DeclareLanguageMapping{american}{american-apa}
\DeclareFieldFormat{apacase}{#1}

\usepackage[T1]{fontenc}
\usepackage{csquotes}

\addbibresource{/home/johannes/Dropbox/references.bib}
\addbibresource{/home/johannes/Dropbox/references2.bib}

\usepackage{floatrow}

\usepackage{listings}
\usepackage{xcolor}
\usepackage{colortbl}

\lstset{
  language=R,                    
  basicstyle=\footnotesize,      
  numbers=left,                  
  numberstyle=\tiny\color{gray}, 
  stepnumber=1,                  
  numbersep=5pt,                 
  backgroundcolor=\color{white}, 
  showspaces=false,              
  showstringspaces=false,        
  showtabs=false,                
  frame=single,                  
  rulecolor=\color{black},       
  tabsize=2,                     
  captionpos=b,                  
  breaklines=true,               
  breakatwhitespace=false,       
  title=\lstname,                
  keywordstyle=\color{red},     
  commentstyle=\color{blue},  
  stringstyle=\color{violet},     
  escapeinside={\%*}{*)},        
  morekeywords={*,...}           
} 




\usepackage{crimson}
\usepackage{microtype}


\usepackage{fancyhdr}
\usepackage{setspace}
\singlespace
\usepackage{longtable}
\usepackage{subfig}
\usepackage[a4paper, total={18cm, 24cm}]{geometry}

\pagestyle{fancy}
\fancyhf{}
\renewcommand{\headrulewidth}{0pt}
\renewcommand{\maketitle}{}

\usepackage{enumitem}
\setlist[itemize]{topsep=0pt,itemsep=0pt,parsep=0pt,partopsep=0pt}

\usepackage{multicol}
\setlength\multicolsep{0pt}

\usepackage{array}
\usepackage{caption}
\usepackage{graphicx}
\usepackage{siunitx}
\usepackage[normalem]{ulem}
\usepackage{colortbl}
\usepackage{multirow}
\usepackage{hhline}
\usepackage{calc}
\usepackage{tabularx}
\usepackage{threeparttable}
\usepackage{wrapfig}
\usepackage{adjustbox}
\usepackage{hyperref}



\newlist{propertyList}{itemize}{1}
\setlist[propertyList]{
  label=\textbullet,
  noitemsep,
  leftmargin=10pt,
  before=\begin{multicols}{3},
  after=\end{multicols}
  }

\rfoot {\thepage}

\listfiles

\setlength{\parindent}{1.2cm}
\author{Johannes Aengenheyster}
\date{\today}
\title{}
\hypersetup{
 pdfauthor={Johannes Aengenheyster},
 pdftitle={},
 pdfkeywords={},
 pdfsubject={},
 pdfcreator={Emacs 28.2 (Org mode 9.6.6)}, 
 pdflang={English}}
\begin{document}


\section*{Structural predictors of private museum founding}

\subsection*{Abstract}

In the last decades a new organizational population of private museums has seen substantial proliferation.
While multiple hypotheses for the spread of this new form have been raised, systematic analyses of these have been lacking.
In particular, the rise of private museums has been hypothesized to be associated to tax incentives, reductions in government spending, increasing inequality and increasing elite wealth. 
Combining various socio-economic and art field data sources, I conduct quantitative tests of these hypotheses with datasets of 1245, 2514 and 3330 country years using random effects negative binomial regression models.
While I find support for a positive effect of tax incentives, government spending is associated non-monotonically (reverse U-shaped) with private museum founding (not, as hypothesized, negatively). 
Furthermore the effects of inequality are divergent, as a positive association with private museum founding is found for wealth inequality and a negative one for income inequality.
Finally, the effect of elite wealth is primarily small and insignificant. 
I conclude with a call for advancing theoretical elaboration and measurement precision to further investigate founding determinants. 





\subsection*{Introduction}

Private museums constitute a substantial development in the cultural sector in the last decades.
While museums have been privately founded throughout history, in recent decades hundreds of private museums have been founded; this increase in their numbers is often referred to as the "private museum boom" \parencite{Walker_2019_collector}. 
These institutions, often established by extremely wealthy art collectors, have both attracted attention and aroused controversy.
Evaluations can range from praise of the public-spiritedness of the founders who expend large sums of their private wealth for the provision of public goods over questions of the sustainability and quality of these institutions to criticisms of the unaccountability and status-seeking behavior of their founders. 
Just as their present behavior and its consequences have aroused curiosity, so has the question of which factors led to their proliferation in the first place:
Is their founding related to a decline of public institutions which shifts decision-making further into the hands of wealthy, private individuals who use the association of public benefit to legitimate increasing inequalities?



To answer these questions, I investigate in this paper the factors that have been hypothesized to have contributed to the proliferation of private museums over the last decades.
In particular, the structural conditions that have been suggested to have facilitated private museum founding are tax incentives, (lack of) state funding, inequality and elite wealth.
However, previous research has been primarily conducted as case studies of individual museums (sometimes based on fieldwork, but often also only on media reports); systematic comparative research has so far been limited.
Furthermore, by focusing on museums that have been founded, existing research has generally sampled on the dependent variable as limited attention has been given to countries without large private museum populations.
At the same time, substantial interest in the general population of private museum exists, indicated by the reference of many publications to the first systematic account of the overall private museum population \parencite{LarrysList_2015_report}. 

Thus much will be gained from a systematic quantitative test of these different mechanisms.
As the quantitative analysis of private museums is still it its infancy, this paper should be understood as a first exploratory analysis that hopefully will be refined in the future. 
In the remainder of the paper, I review the literature on the causes of private museum founding, and describe the data sources used to investigate each mechanism.
I then report results of regression analyses, which indicate support for a positive effect of tax incentives, an inverse U-shaped relationship with cultural spending, divergent effects of different inequality measures and no support for an effect of elite wealth.



\subsubsection*{Private Museum Definition}

Effectively delineating private museums is not a straightforward task, and depends on the conceptualization of the terms "private" and "museum", neither of which are unambiguous concepts.
For this study I use "private" to denote being established by a private art collector, art collector couple, or their descendants; such a focus on the individual founder thus does not include museums operated by companies (what may more generally be called the "private sector") or groups of people.
A "museum" is conceptualized as an organization that possesses a building and a collection of its own (which excludes kunsthalles and exhibition centers), which is sufficiently accessible to the public.
For the purpose of this research I also only include institutions that focuses on modern or contemporary art, with modern and contemporary art operationalized as art created after 1900. 


\subsection*{Theoretical framework}




\subsubsection*{Tax Incentives}





Tax incentives have been argued to influence influencing various kinds of charitable or philanthropic behavior in general and private museum founding in particular.
Tax incentives decrease the net cost of charitable activities, and can thus be expected to shape philanthropic behavior to the extent to which these activities are not unconditional of economic considerations\footnote{While themes of disinterestedness and intrinsic passion are strongly present in the self-presentations of founders \parencite{Bechtler_Imhof_2018_future,Duron_2020_rebaudengo}, they might constitute strategic framings to adhere to art market norms \parencite{Velthuis_2007_talking} rather than real motivations guiding actual behavior.}. 
Investigating charitable donations more generally, \textcite{Peloza_Steel_2005_elasticities} find in a meta-analysis (primarily of individual behavior in the US) a price-elasticity of donations, i.e. a stronger tendency to make charitable contributions in the presence of financial incentives.






Tax incentives, such as exemption of charitable organizations from some taxes and tax deductibility of donations to such organizations, have also been argued to have contribute to the founding of private museums.
\textcite{Walker_2019_collector} investigates the legal frameworks and tax exemptions that private museums receive in a number of locations.
In the case of Australia, she argues that tax reforms which increased tax deductions for donations to charitable organizations played a substantial role in the establishment of private museums: "The founding and funding of TWMA [TarraWarra Museum of Art] was a direct consequence of the changes to Australian tax law. [\ldots{}] without [a broader range of inducements and approaches], I would suggest a generosity of spirit alone will not result in a substantial philanthropic gift such as TWMA or the founding of other private museums in Australia."
In the US, the tax code is also argued to "[support] the founding of collector museums" (p.35) through incentives such as exemptions from capital gains tax and tax deductibility of exhibition, insurance, conservation and storage costs.


For Europe, the effect of tax incentives on founding is less pronounced.
While tax incentives in Germany (p.29) and Switzerland (p.36) are discussed, it is not argued that they constitute a necessary or supportive factor in the establishment of private museums, it is merely mentioned that "[their] complex tax structure is a way for [a] foundation to minimize its tax burden." (p.36).
Nevertheless, a clear overall argument is reached regarding the impact of tax incentives as it is concluded that "the formation of many private museums in the last two decades are directly or indirectly linked to generous tax incentives as they help to offset individual tax burdens and promote arts philanthropy. [\ldots{}] The generosity and breadth of American, Australian, English and European tax law is a factor in the promotion of private museums" (p.37). 






The impact of tax incentives does not exclude other motivations, thus arguments which highlight other motivations such as "passion for art" (\cite[p.7]{Zorloni_Resch_2016_opportunities}, \cite[p.12]{Adam_2021_rise}) or other "personal motivations" \parencite[p.144]{Walker_2019_collector} do not contradict a facilitative role of tax incentives. 
In fact, other motivations (such as the aforementioned passion for the arts, as well as potential reputation gains; cf. \cite{Bekkers_Wiepking_2010_literature,Bekkers_Wiepking_2011_philanthropic} for motivations of philanthropic behavior generally) are required as tax incentives only \emph{reduce} the cost of philanthropic behavior but do not cancel it out completely, let alone produce a financial profit; tax incentives thus may play a supportive role by easing the financial burdens of running a private museum as a consequence of other motivations.


However, an different reason why tax incentives might not have noticeable effects might be limited knowledge of and familiarity with them among potential beneficiaries (collectors deliberating the choice of opening a museum), a mechanism which has been argued to negatively affect private museum founding in China \parencite[p.222]{DeNigris_2018_museums} and charitable donations in Europe generally \parencite{Hemels_2017_incentives}.
While lack of familiarity might prevent an effect of existing tax incentives, it seems unlikely to be the case here as the topic of tax incentives has been widely covered in publications both scholarly \parencite{Walker_2019_collector,Reich_2018_philantropy,Zorloni_Resch_2016_opportunities} and journalistic (e.g. \cite{Brown_Pes_2018_taxpayers,Boucher_2020_solow,Honig_2016_IRS}).
Furthermore, the decision to open a private museum is likely receiving more deliberation than smaller acts of philanthropy (such as donations) due to the size of the necessary planning efforts (which sometimes results in the establishment of separate foundations staffed by professionals); it is hence unlikely that during this planning process financing pathways through tax incentives are not explored.


Therefore, it seems plausible to assume an influence of tax incentives, leading to the following hypothesis:

\bigbreak
\noindent
\textbf{Hypothesis 1}: Tax incentives are associated with higher rates of private museum founding.




\subsubsection*{State funding}






Another argument for the emergence of private museums has been the decline or absence of state support for the arts. 
Whereas in the mid-to-late 20th century "wealthy industrialist collectors [\ldots{}] looked to the public museum as a suitable home for their collections" \parencite[p.144]{Walker_2019_collector}, declines in cultural spending are argued to have reduced public museum capacities, thus making them less attractive for conserving, displaying and storing donations of private collectors, which in turn leads "many collectors [to] feel that they are better qualified to create their own museums" (ibid, p.145).
While some forms of public-private partnerships "[appear] to contradict the premise that many private museums are founded because of the weakness of the public sector and the limitations of the public museum" (p.150),  \textcite{Walker_2019_collector} nevertheless argues that "the growing popularity of German private collector museums can be attributed to the difficult economic climate that is negatively impacting public museums" (p.146). 
Similarly, \textcite{Boloten_Hacking_2021_foreword} argue that "with their inevitably strained resources, public museums often struggle to satisfy the substantial demands of the collector/patron communities on which their collections, and institutions, very much rely. Against this backdrop [\ldots{}] [art collectors] are increasingly turning to bespoke Private Museum solutions" (p.10). 
In particular, limited exhibition space in public museums has been argued to have been contributed to the rise of private museums (\cite[p.217]{Walker_2019_collector}, \cite[p.1]{Zorloni_Resch_2016_opportunities}). 
Another pathway between low government spending and relatively high private museum founding concerns emerging art fields in developing countries, as it is argued that private individuals are more likely to dedicate philanthropic efforts to the arts there than in places where the cultural sector is already supported by the state \parencite{Durand_2018_jumex,Bechtler_Imhof_2018_future,Boloten_Hacking_2021_foreword}, a pattern that could correspond to the founding of gilded age cultural institutions by industrialists in the late 19th/early 20th century \parencite{diMaggio_1982_boston,Adam_2004_philanthropy}. 







Such theorized substitutive relations between state and private funding are more generally referred to as "crowding-out" in philanthropic studies (cf. \cite{Bekkers_Wiepking_2010_literature} for a literature review;  the lack of engagement with the term and related studies thus also indicates the early state of theoretical development of the private museum literature).
The term "crowding-out" originated in response to the question of how charitable behavior would change if states were to \emph{increase} funding, thus potentially "crowding out" private donors.
While the situation is a different one here, with private museum founding being related to \emph{decreasing} state funding (or the absence of it), the underlying theoretical mechanism is the same:
In both cases, donors are motivated (for altruistic or other reasons) to provide public goods, and thus allocate their donations in response to the funding choices of the government.
However, \textcite{DeWit_etal_2018_philanthropy} and \textcite{Lena_2019_entitled} argue that government funding could also signal valuation of an activity, which raises the possibility of a positive association between government funding and private museum founding ("crowding-in").
Nevertheless, as the crowding-in hypothesis has not been employed in the case of private museums, the hypothesized effect of government funding in the private museum literature can be formulated as follows:  

\bigbreak
\noindent
\textbf{Hypothesis 2}: Higher government cultural spending is associated with lower rates of private museum founding.


\subsubsection*{Inequality}


Inequalities have been suggested to be related to the proliferation of private museum foundings as well. 
\textcite{Brown_2019_private} points out parallels to previous periods of wealth accumulation, arguing that "the founding of private museums appeals to the rhetoric of social obligation that sustained the accumulation of capital early in the twentieth century and that induced publics not to object to the rise to power of wealthy classes" (p.15).
Current times are argued to resemble again the gilded age of the late 19th and early 20th century where "acts of philanthropic giving [\ldots{}] functioned as a means by which to maintain widespread confidence in a socio-economic system that enabled a small social cadre to accrue long-term financial advantage" (p.3). 
Rising inequality is argued to create a legitimacy deficit, to which the founding of private museums provides a not a substantial solution, but a distraction which despite being superficial seems to be effective nonetheless.



While only \textcite{Brown_2019_private}  makes an explicit argument for a relationship between inequalities and private museum founding (as other mentions of inequality, such as \textcite{KalbCosmo_2020_museum} or \textcite{Adam_2021_rise} do not raise a specific relationship between inequality and private museum founding), a literature stream that might be best summarized as "critical philanthropy studies" foregrounds links between philanthropy more generally and distributional issues:
For example, \textcite{Maclean_etal_2021_philanthropy} argue that "the ultimate purpose of elite philanthropy [\ldots{}] is to legitimate and make palatable the extreme inequalities generated by the forward march of global capitalism" (p.14)
A similar arguments is made by \textcite{Glucksberg_RussellPrywata_2020_philanthropy}, who compare the philanthropic donations of the most charitable givers to their business activities and, after finding the former to be dwarfed by the latter, argue that "philanthropy plays a role in helping elites legitimize their own wealth, and thus in legitimizing inequality" (p.2).
In the same vein, \textcite{Giridharadas_2018_winners} argues that inequality causes dissatisfaction, leading philanthropists to engage in philanthropy "out of a mix of altruism and the self-preservational desire to cool public anger" (p.172), a point also shared by \textcite{KohlArenas_2015_selfhelp} who argues that "private foundations and social movement organizations construct idealized spaces of public participation and discourse theories of change that draw attention away from structural inequality" (p.796). 
As these scholars unanimously posit a positive association between inequality and philanthropy, the hypothesis can be formulated as: 

\bigbreak
\noindent
\textbf{Hypothesis 3}: Higher income and/or wealth inequality is associated with higher rates of private museum founding.



\subsubsection*{Elite Wealth}


The founding of private museums has been associated with the rise of (Very or Ultra) High Net Worth Individuals (HNWI, VHNWI, UHNWI; corresponding to a wealth of 1 million, 5 million and 30 million USD respectively).
\textcite{Gnyp_2015_collectors} notes that a "reason for the rise of private museums could be the accumulation
of wealth of the UHNWIs, whose income and numbers have been growing" (p.145).
Similarly, \textcite{Zorloni_Resch_2016_opportunities} argue that "due to [\ldots{}] the increasing number of HNWIs, the number of private museums has been increasing in recent years" (p.12).
Less explicit about a connection between elite wealth and private museum founding, \textcite{Walker_2019_collector} nevertheless mentions "rapid increases in [\ldots{}] private wealth" (p.145) as one of the conditions that strengthened the position of collectors vis-a-vis public museums, presumably as wealth increases allowed them to set up their private museums. 
Since "such private spaces cost money and often do not generate any income" \parencite[p.145]{Gnyp_2015_collectors}, they are only feasible to maintain by individuals with substantial disposable funds. 



However, there have also been calls for caution against an exclusive focus on "super rich collectors who attract so much attention [as] there are many others that have made and still make things possible with comparatively small means" \parencite[p.12]{Bechtler_Imhof_2018_future}.
Nevertheless, even these "comparatively small means" are presumably comparatively small only in relation to the absolutely richest individuals (e.g. billionaires), and are thus likely still located in the range of what would be classified as HNWIs.
A hypothesis on elite wealth can thus be formulated as: 

\bigbreak
\noindent
\textbf{Hypothesis 4}: Higher numbers of (V/U)HNWIs are associated with higher rates of private museum founding.




\subsection*{Data}


\begin{table}[htbp]
\caption{\label{tbl:data_srcs}Concepts, operationalizations and data sources}
\centering
\begin{tabular}{lp{7cm}l}
\hline
Concept & Indicator & Data source\\
\hline
Private Museum Fouding & Number of PM openings & PM Database\\
Tax Incentives & Tax deductibility of donations & Charities Aid Foundation\\
 & Top Marginal Income Tax Rates & Fraser Institute\\
Cultural spending & Total government expenditure on Culture, Recreation and Religion (COFOG 08) & OECD, UN, Eurostat, IMF\\
Inequality & Wealth/Income inequality: 10\% share, 1\% share, Gini & World Inequality Database\\
Elite Wealth & Population above threshold of 1M, 5M, 30M, 200M USD & World Inequality Database\\
Size & Population & World Bank\\
Development/art demand & GDP per capita & World Bank\\
Presence of potential founders & Number of art collectors & Artnews top200 collector ranking\\
Museum population & Number of modern/contemp. art museums & Museums of the World Database\\
Density: Legitimacy & Number of PMs open & PM Database\\
Density: Competition & Number of PMs open\textsuperscript{2} & PM Database\\
\hline
\end{tabular}
\end{table}


Ideally collector-level data would be used to construct collector-year as the unit of analysis, which would allow the investigation of collector decision-making by utilizing collector-level variables. 
However, no such databases exist of the art collector population, in particular of its longitudinal development, which makes such ideal analysis unfeasible. 
Therefore the more aggregate unit of country-years is used as the unit of analysis (with countries defined according to the ISO 3166 standard). 
Data sources are summarized in Table \ref{tbl:data_srcs} and described below.
Finally, a number of additional data processing steps are described in appendix \ref{app_data_processing}.


\subsubsection*{Dependent Variable: Number of private museum foundings per country-year}


To document the development of private museums, existing databases of private contemporary art museums \parencite{LarrysList_2015_report,Independent_collectors,global_private_museum_network_2020_museums,BMW_Independent_Collectors_2018_artguide,vdEerenbeemt_vdWauw_DDD_2016} have been combined.
Additionally, web research has been conducted by searching in a number of online art media\footnote{The media sources used were: Artforum, Artnet, Art Territory, Artsy, My Art Guides, Artnews, Artfcity, Frieze, The Art Wolf, The Art Newspaper, Art Privee, Widewalls, White Hot Magazine and Hyperallergic.} for the term "private museum", "private art museum" and "private contemporary art museum".
After recording the names and deleting duplicates, it was investigated using publicly available information whether the organization corresponds to the employed definition of a private museum.
Following this, additional information about each museum was collected by student assistants, the most relevant for this research being country and opening year.
Currently, the database includes 547 museums (of which 446 are currently open, the remainder having closed or transformed to the extent that they no longer constitute a private museum), which are located in 64 countries, with the majority being located in Western Europe (136), East Asia (97) and North America (77).
The majority of them (420, or 76.8\%) have been founded after 2000.
Information on country and opening year is used to construct a count indicator of the number of private museums opened in each country-year, the typical operationalization for studying organizational founding \parencite{Bogaert_etal_2014_ecological}. 
This study has a global scope and includes a large number of countries without private museums to avoid sampling on the dependent variable. 
However both the time period and number of countries that can be investigated are limited by the coverage of the independent variables. 


\subsubsection*{Independent Variables}

\paragraph*{Tax incentives}

For this study, I focus on top marginal income tax rates, i.e. the amount of tax to be paid on income in the uppermost tax bracket, and tax deductibility of donations.
Higher top marginal income tax rates (in combination with tax deductibility of charitable donations) generally decrease the net cost of a donation (as higher rates increase the amount of taxes returned; cf. \cite{Hemels_2017_incentives} for a more detailed discussion), and are thus expected to increase the incentive for founders to donate their collection (or parts of it) to their private museum and receive a tax deduction in return. 


Top marginal income tax rate data is taken from the Index of Economic Freedom of the World \parencite{Fraser_2022_economic_freedom}, of which it constitutes a sub-item.
Tax deductibility of donations is measured from the report "Rules to Give - A Global Philanthropy Legal Environment Index" \parencite{Quick_Kruse_Pickering_2014_philanthropy}.
I use the binary indicator measuring whether individuals can receive tax deductions for donations to non-profits\footnote{The report also includes a binary indicator whether non-profit organizations are exempt from at least some taxes, but as this is the case in the vast majority of countries, it does not provide sufficient variation to serve as an explantory variable.}.
The report describes the situtation in 193 countries but only covers the state of the legal environment at its publication date, therefore tax deductibility of donations can only be investigated in regards to between-country differences.
As top marginal income tax rates are only expected to provide an incentive if tax donations are tax deductible, I calculate an interaction term between the tax deductibility of donations and marginal income tax rates. 

\paragraph*{Cultural Spending}


I use data on government spending on culture, recreation and religion collected by the UN \parencite{UN_2022_consumption}, the IMF \parencite{IMF_2022_GFS}, the OECD \parencite{OECD_2022_SNA_TABLE11_ARCHIVE,OECD_2022_SNA_TABLE11} and Eurostat \parencite{Eurostat_2022_COFOG} as an indicator of cultural spending.
These organizations use the System of National Accounts (SNA), where Classification of the function of government (COFOG) Code 8 describes "culture, recreation and religion".
The single digit code 8 is rather broad, and the more detailed  double digit SNA code 82 "cultural services" would be preferable to exclude state spending on e.g. sport events, broadcasting, publishing and religion. 
However, data coverage is substantially worse for SNA Code 82, and as there appears to be substantial correlation between Codes 8 and 82 for countries that provide data on both levels, I consider it acceptable to use the broader single-digit Code 8.
Even with the broader single-digit Code 8, cultural spending constitutes the indicators with the smallest coverage; I therefore combine as multiple data sources (the exact combination procedure is described in Appendix \ref{app_data_coverage}, also see figure \ref{fig:vrbl_cycnt}).


\paragraph*{Inequality}

The WID provides Gini coefficients for wealth and pre-tax income\footnote{While data on post-tax income is also available in the WID, its coverage is very limited}.
The top 1\% and 10\% wealth and income shares are used as alternative measures of inequality (the actual process of operationalizing concepts with different measures is described in section Variable selection).


\paragraph*{High Net Worth Individuals}

I use the WID to calculate HNWIs indicators, i.e. the population above a certain wealth threshold in a given year. 
The WID contains for each country-year measures of wealth at various quantile thresholds, e.g. the value for the 90th percentile describes the amount of wealth of the least wealthy member of the top 10\%.
With higher quantiles, higher resolutions are included: Up to the 99th percentile, the largest step is 1\% (e.g. 91st,92nd, etc. quantile), while the 99th percentile is subdivided further into percentile tenths (99.1,99.2, etc.).
The top 0.1\% and 0.01\% (the 99.9th and 99.99th quantile) are similarly subdivided into ten smaller slices.
Amounts are reported in local currencies and are therefore converted into 2021 USD using market exchange rates provided by the WID.
I use these wealth thresholds to calculate the percentage of people with a wealth of at least 1 million, 5 million, 30 million, 200 million USD\footnote{Despite large number of thresholds and higher resolution of higher percentiles, I do not consider it adequate to measure numbers at thresholds higher than 200M.} by linearly interpolating between the two nearest percentiles above and below the respective threshold.
If the threshold exceeds the value of the highest quantile, I assign a zero.\footnote{Linear interpolation likely overestimates the percentage of HNWIs if a higher threshold exists (as the wealth distribution probably does not increase linearly), and underestimates it when no higher threshold exists (as necessarily 0 is assigned). I argue that given the high number of thresholds this procedure still produces useful approximations.}
The so-obtained proportion of HNWIs is converted into a count using World Bank population size data.


\subsubsection*{Control Variables}

\paragraph*{GDP and Population size}

GDP is included to account for potentially imperfect measurement of government spending and other art field indicators (discussed below). 
Furthermore, given the stratified character of art consumption \parencite{Bourdieu_1984_distinction} within countries, controlling for GDP also includes a demand dimension as more developed countries potentially host larger population shares interested in post-material art consumption.
Population size is included as an offset (cf. section Regression specification). 



\paragraph*{Artnews collector ranking}

As private museums are founded by collectors, they are more likely to be founded in country-years where many potential founders exist.
While a complete measure of the art collector population is not feasible (which necessitated the choice of country-year instead of collector-year as the unit of analysis), it is still possible to construct a proxy. 
For this purpose I construct an indicator of the population of art collectors from the Artnews magazine collector ranking, an index published yearly since 1990 of the 200 art collectors the magazine considers most important.
As Artnews includes the country of residence of each collector, I construct the collector population indicator for a country-year as the number of collectors included in the Artnews ranking for that country-year.
Artnews furthermore includes information on the genre(s) of each collector, which allows me to only include collectors who collect modern and contemporary art.
I argue that the number of Artnews collectors per country-year constitutes an indicator of the wider propensity of art collecting in a country (rather than a pool of individuals able to establish a museum) and therefore do not remove art collectors from the index after they have founded a museum (which concerns 141 collectors).



\paragraph*{Density dependence}

Research on organizational population \parencite{Carroll_1989_density,Hannan_1992_dynamics} has argued that foundings of organization are driven by legitimacy and competition:
While increasing legitimacy of a new organizational form is argued to encourage new foundings, competition is hypothesized to discourage it.
Both legitimacy and competition are generally derived from the number of already existing organizations, which is referred to as "density", leading to the label of density-dependence for this paradigm. 


However, density effects have been argued to be weaker for populations of non-profit organizations \parencite{Bogaert_etal_2014_ecological}, as these are characterized by demands more diverse and conflicting than the for-profit companies for which the density dependence paradigm was originally developed; in particular the absence of the profit motive is argued to correspond to a weaker legitimation effect. 
Nevertheless, private museums may still compete over audiences \parencite[p.14]{Adam_2021_rise}, which \textcite{Frey_Meier_2002_beyeler} argue are valued both economically (through ticket prices) and symbolically (as validating exhibition choices).
Furthermore, the marginal value of a private museum as an instrument of distinction might decrease as more and more founders establish their museums; in this view, the symbolic capital to be gained by private museum founding constitutes the objective to compete for. 



I therefore include for each country-year measures based on the number of private museums open at that year, in particular a linear term (which measures legitimacy) and a squared term (which measures competition).
Given the transnational nature of the art field (due to which legitimation and competition might not be limited to country borders) I also add global linear and squared density measures which are the same for every country in a given year and describe the number of museums open worldwide in a given year.


Another variable of the organizational population concerns closures:
The private museum population has seen substantial numbers of closures with 101 institutions having been shut down or transformed (cf. \cite{Velthuis_Gera_forthcoming_fragility}), and growing awareness of the fragility of these institutions might in turn make their founding less attractive to founders keen of leaving a mark in history.
I therefore add a global variable of the number of cumulative museum closures to capture the potential delegitimating effect of the closing of institutions that often are planned as personal legacies \parencite{Walker_2019_collector}.\footnote{While one could measure this variable also on a country-level, the low number of closings would lead to a highly skewed variable, and therefore to coefficients disproportionately reflecting a few countries (in particular Switzerland, the United States and Netherlands, depending on whether the estimator measures closing numbers or proporitions). I therefore limit myself to measuring the global effect of museum closures.} 




\paragraph*{Museum population}


Private museums populations might interact with other museum populations in a manner similar to the density paradigm. 
On the one hand, existing museums are indicative of familiarity and legitimacy in a country with museums as such, and then likely also indicate that museums are being valued, thereby possibly raising the reputational gains private collectors anticipate from founding a museum of their own, and thus the likelihood of founding one.
An existing museum population might also facilitate private museum founding by providing a pool of personnel with skills and contacts that can be recruited \parencite{Quemin_2020_power}.
However, existing museums may also compete with private museums for audiences, and potentially symbolic capital (for example, in countries with large public museum populations private collectors might have to invest substantial amounts to achieve the quality of public museums, thus discouraging all but the wealthiest private founders). 


I thus use the Museums of the World database \parencite{deGruyter_2021_MOW} to describe a country's art museum population. 
As preliminary analyses indicated that the database has not been maintained in recent years (due to a strong decline in museum foundings being covered) I use it only to construct time-invariant indicators of the number of non-private modern and contemporary art museums having opened until 1990.
In order to measure both legitimating and competitive effects, I include a linear and squared terms as in the private museum density measures.



\subsubsection*{Data processing}

A number of observations have been excluded for various reasons:
The exchange rates for Zimbabwe and Venezuela were deemed unrealistic (Venezuela is discussed by \textcite{Blanchet_2017_conversions}); both countries were therefore excluded.
The HNWI indicators for the United Arab Emirates, Saudi Arabia and Qatar display a strong decline between 2000 and 2010 of which both the starting value and the strength of the decline are extremely high (starting at several standard deviations above the mean and declining to average values) and hence likely caused by data issues; the countries are therefore excluded. 
Furthermore, a number of country-years have been removed for Yemen due to negative cultural spending.
For a number of years the wealth gini coefficient of South Africa was larger than 1; it has therefore been set to a ceiling of 0.95.
Finally, Iceland, the Bahamas, Monaco and Liechtenstein have been excluded as these countries' small population results in an extremely high rate of Artnews top 200 collectors per capita (and in the case of Iceland, also an extremely high rate of modern/contemporary art museums in 1990).
Since the number of Artnews collectors is a discrete count variable it is unable to provide accurate measures in countries with very small populations which justifies the exclusion of these countries on methodological grounds.




\subsection*{Analytical Strategy}


During data collection a number of challenges became apparent:
First, data coverage on some variables was substantially limited; second, the previous literature was not specific enough to derive a single precise measurement of some concepts, namely inequality and HNWIs; and third, the temporal duration of the mechanisms has not been investigated in the literature. 
I address these challenges as follows: 

\subsubsection*{Variable combinations}

Despite substantial efforts to collect complete data for the relevant variables, it was not possible to collect data for all country-years; especially cultural spending and top marginal income tax rate have considerable coverage gaps.
As excluding a large number of country-years limits the generalizability of findings, I construct multiple datasets that differ regarding variable and country-year coverage.
In particular, I construct one dataset with all variables (1245 country years, 85 countries; referred to as "DS all IVs"), one with all variables except cultural spending (2514 country years, 150 countries; "DS --CuSp"), and one with all variables except cultural spending and top marginal income tax rates (3330 country years, 161 countries; "DS --CuSp/TMITR").
Most country-years are situated between 2000 and 2020 with coverage generally increasing over time (figure \ref{fig:cbn_cycnt}); the datasets therefore effectively cover the period in which the majority of private museum founding has taken place (they cover 64.9\%, 73.5\% and 73.9\% of foundings, respectively). 
\begin{landscape}
% latex table generated in R 4.3.0 by xtable 1.8-4 package
% Wed Jul  5 15:56:59 2023
\begin{table}[ht]
\centering
\begin{tabular}{p{7cm}llllllllllll}
  \hline 
  & \multicolumn{4}{c}{DS all IVs (n=1245)} & \multicolumn{4}{c}{DS --CuSp (n=2514)} & \multicolumn{4}{c}{DS --CuSp/TMITR (n=3330)} \\ 
\cmidrule(r){2-5}\cmidrule(r){6-9}\cmidrule(r){10-13}  
 Variable & Mean & SD & Min. & Max. & Mean & SD & Min. & Max. & Mean & SD & Min. & Max. \\ 
  \hline
Private Museum openings & 0.011 & 0.054 & 0 & 0.881 & 0.006 & 0.039 & 0 & 0.881 & 0.005 & 0.034 & 0 & 0.881 \\ 
  Tax deductibility of donations & 0.906 & 0.292 & 0 & 1 & 0.774 & 0.419 & 0 & 1 & 0.682 & 0.466 & 0 & 1 \\ 
  Top Marginal Income Tax Rate (\%) & 35.2 & 14.0 & 0 & 63.5 & 32.2 & 13.5 & 0 & 65 & -- & -- & -- & -- \\ 
  Gvt cultural spending (millions) & 344 & 312 & 0.292 & 2 021 & -- & -- & -- & -- & -- & -- & -- & -- \\ 
  Income share of top 10\% (*100) & 40.4 & 8.88 & 26.5 & 69.6 & 45.5 & 9.58 & 26.5 & 70.8 & 45.9 & 9.14 & 26.5 & 71.6 \\ 
  Income share of top 1\% (*100) & 14.0 & 4.47 & 6.04 & 29.7 & 16.3 & 5.32 & 6.04 & 44.9 & 16.2 & 5.23 & 6.04 & 44.9 \\ 
  Gini of pre-tax income (*100) & 51.5 & 8.62 & 35.9 & 78.1 & 56.4 & 9.05 & 35.9 & 78.1 & 56.9 & 8.55 & 35.9 & 78.1 \\ 
  Wealth share of top 10\% (*100) & 60.6 & 7.54 & 42.0 & 90.9 & 63.2 & 7.87 & 42.0 & 90.9 & 63.3 & 7.69 & 42.0 & 90.9 \\ 
  Wealth share of top 1\% (*100) & 27.2 & 7.77 & 12.1 & 58.2 & 30.0 & 8.24 & 12.1 & 58.2 & 30.0 & 8.07 & 12.1 & 58.2 \\ 
  Gini of net wealth (*100) & 75.6 & 6.19 & 57.7 & 95 & 77.4 & 6.30 & 57.7 & 95 & 77.4 & 6.12 & 57.7 & 95 \\ 
  HNWIs (net worth $\geq$ 1M) & 14 517 & 20 978 & 16.4 & 142 766 & 7 857 & 16 502 & 13.5 & 142 766 & 6 035 & 14 704 & 0 & 142 766 \\ 
  HNWIs (net worth $\geq$ 5M) & 986 & 1 622 & 0 & 12 420 & 545 & 1 245 & 0 & 12 420 & 421 & 1 105 & 0 & 12 420 \\ 
  HNWIs (net worth $\geq$ 30M) & 54.7 & 109 & 0 & 966 & 30.6 & 81.4 & 0 & 966 & 23.7 & 71.9 & 0 & 966 \\ 
  HNWIs (net worth $\geq$ 200M) & 5.73 & 9.80 & 0 & 80.4 & 3.12 & 7.61 & 0 & 80.4 & 2.39 & 6.76 & 0 & 80.4 \\ 
  GDP per cap. (thousands) & 26.4 & 24.9 & 0.453 & 135 & 16.2 & 21.1 & 0.277 & 135 & 13.1 & 19.4 & 0.247 & 135 \\ 
  Population (millions) & 65.9 & 202 & 0.087 & 1 411 & 53.5 & 169 & 0.087 & 1 411 & 43.0 & 148 & 0.081 & 1 411 \\ 
  modern/contemp. art museums (1990) & 0.730 & 1.01 & 0 & 5.21 & 0.413 & 0.817 & 0 & 5.21 & 0.324 & 0.734 & 0 & 5.21 \\ 
  Collectors in Artnews top 200 collector list & 0.069 & 0.247 & 0 & 2.23 & 0.036 & 0.178 & 0 & 2.23 & 0.027 & 0.155 & 0 & 2.23 \\ 
  PM density (country) & 0.170 & 0.279 & 0 & 1.76 & 0.091 & 0.214 & 0 & 1.76 & 0.069 & 0.190 & 0 & 1.76 \\ 
  PM density (global) & 294 & 107 & 68 & 429 & 300 & 107 & 68 & 429 & 282 & 109 & 68 & 429 \\ 
  Nbr. PM closings (cumulative, global) & 22.6 & 23.5 & 1 & 75 & 24.2 & 24.3 & 1 & 75 & 20.6 & 23.2 & 1 & 75 \\ 
   \hline \multicolumn{13}{l}{\footnotesize{all country-level count variables are per million population; all monetary amounts are 2021 USD}}
\end{tabular}
\caption{Summary Statistics} 
\label{tbl:descs}
\end{table}
\end{landscape}


\subsubsection*{Variable selection}
\label{vrbl_slctn}
While theoretical relations involving the number of HNWIs and inequality have been proposed, concrete operationalizations have been missing.
For the number of HNWIs, I randomly choose a threshold of 1, 5, 30, 200 million USD, while inequality is first differentiated into wealth and income inequality, for which in turn one of Gini, 10\% share and 1\% share is chosen randomly.
This choice results in 36 possible variable combinations (4 HNWI variables \texttimes{} 3 income inequality variables \texttimes{} 3 wealth inequality variables). 


\subsubsection*{Time lags}

So far there has been limited investigation into the time frames each mechanism needs to take effect.
For example, a decrease of cultural spending might not become effective immediately as it might take time for collectors to perceive this change repspond accordingly.
As a misspecified time lag might miss an effect that exists at a different time-lag, I take an exploratory approach and vary the lag of each longitudinal variable between one and five years.
To obtain comparability between the models when using different time-lags, I limit the country-years to those that have data on all of the five preceding years for all time-lag variations.
As a complete exploration of the lag-choice space is not possible (it grows exponentially with the number of longitudinal variables included), I use an optimization procedure:
For a selection of variables, I first select random starting values for the lags.
I then select all variables in random order, and for each variable in turn vary the lag by calculating five regression models (see Section Regression specification) identical except for the lag of the variable in question, and keep the best fitting one (characterized by maximum log-likelihood), the lag of which is then used when optimizing the next (randomly chosen) variable.
Lags of squared variables and interactions are fixed to the lag of their main variables.
If multiple models fit equally well, a lag among these is selected at random.
This selection of variables in random order followed by lag choice is continued until no further improvement is achieved.
While the random starting values and random order of variables help to explore the lag space, models can still be stuck in local optima; I therefore run each model four times, each with a different set of random starting values (see figure \ref{fig:lag_dens}). 



\subsubsection*{Regression specification}

Each regression model is specified as a negative binomial model (which is robust to potential overdispersion of the dependent variable) run using the glmmTMB command \parencite{Brooks_etal_2017_glmmTMB} of R 4.3.0. 
Random country intercepts are added to account for clustering of observations in countries.
I furthermore use population as an offset (which estimates founding rate per capita) to allow comparison between countries of different population sizes; all count predictors are therefore first transformed into per capita rates
This results in the following model specification:

\begin{equation*}
y_{it} = \beta_i + \beta X + 1 \times ln(pop_{it}) + \epsilon_{it}
\end{equation*}

with \(y_{it}\) as number of private museum foundings in country \(i\) for year \(t\), \(\beta_i\) as country-intercept, \(\beta\) as coefficient vector of predictor matrix \(X\), \(1 \times ln(pop_{it})\) as the population offset, and \(e_{it}\) as error component.
To facilitate convergence and interpretation, all variables except the binary variable of tax deductibility of charitable donations are also rescaled to a mean of 0 and a standard deviation of 1. 
In total 137 734 models are run, all of which converge successfully. 
Given 36 variable combinations which are optimized five times with different starting values each per three datasets this results in an average of 320 regression models per variable combination.


\subsection*{Results}


\subsubsection*{Descriptive findings}


Summary statistics for all variables are provided in unscaled form in table \ref{tbl:descs}, providing two relevant insights. 
Firstly, it can be seen that the smaller datasets, especially "DS all IVs", where countries of the Global North are overrepresented, are characterized disproportionately by higher rates of HNWIs, lower inequalities, higher GDP per capita and larger populations of museums and collectors of modern and contemporary art.
The main variable of interest, private museum founding (and consequently private museum densities), is more prevalent in "DS all IVs" as well: 
For example, while in "DS all IVs" on average each year one private museum is founded per 89.6 million inhabitants (1/0.0112), this rate is substantially lower in "DS --CuSP" and "DS --CuSp/TMITR" with one private museum founding per 166 and 216 million inhabitants, respectively.
Secondly, several variables show substantial skew (indicated by high standard deviations compared to the mean), such as cultural spending, GDP and HNWI, art museum and collector rates, as well as the dependent variable of private museum openings: 
Only 211 country-years see show at least one private museum founding in "DS all IVs" (16.9\%), while the other datasets see even lower proportions (10.0\% for "DS --CuSP" and 7.6\% for "DS --CuSP/TMTIR", respectively).



\subsubsection*{Regression results}


% latex table generated in R 4.3.0 by xtable 1.8-4 package
% Wed Jul  5 15:56:59 2023
\begin{table}[ht]
\centering
\begin{tabular}{p{0.5mm}p{6cm}D{)}{)}{8)3} D{)}{)}{8)3} D{)}{)}{8)3} }
  \hline 
  &   & \multicolumn{1}{c}{DS all IVs} & \multicolumn{1}{c}{DS --CuSp} & \multicolumn{1}{c}{DS --CuSp/TMITR} \\ 
  \hline
 & Intercept & -4.49 \; (0.60)^{***} & -6.61 \; (0.52)^{***} & -6.98 \; (0.45)^{***} \\ 
   \multicolumn{5}{l}{\textbf{H1: Tax incentives}} \\ 
 & Tax deductibility of donations & .15 \; (0.58) & .56 \; (0.52) & .72 \; (0.44) \\ 
   & Top Marginal Income Tax Rate & -.62 \; (0.31)^{*} & -.47 \; (0.37) &  \\ 
   & Tax deductibility *
Top Marg. Inc. TR & .66 \; (0.33)^{*} & .84 \; (0.40)^{*} &  \\ 
   \multicolumn{5}{l}{\textbf{H2: Cultural spending}} \\ 
 & Gvt cultural spending & .38 \; (0.15)^{**} &  &  \\ 
   & Gvt cultural spending (squared) & -.63 \; (0.15)^{***} &  &  \\ 
   \multicolumn{5}{l}{\textbf{H3: Income inequality}} \\ 
 & Income share of top 10\% & -.47 \; (0.19)^{*} & -.84 \; (0.22)^{***} & -.82 \; (0.21)^{***} \\ 
   & Income share of top 1\% & -.57 \; (0.13)^{***} & -.69 \; (0.20)^{***} & -.72 \; (0.19)^{***} \\ 
   & Gini of pre-tax income & -.21 \; (0.18) & -.55 \; (0.20)^{**} & -.59 \; (0.20)^{**} \\ 
   \multicolumn{5}{l}{\textbf{H3: Wealth inequality}} \\ 
 & Wealth share of top 10\% & .33 \; (0.10)^{**} & .54 \; (0.15)^{***} & .53 \; (0.15)^{***} \\ 
   & Wealth share of top 1\% & .42 \; (0.12)^{***} & .69 \; (0.18)^{***} & .69 \; (0.17)^{***} \\ 
   & Gini of net wealth & .13 \; (0.11) & .34 \; (0.13)^{**} & .33 \; (0.12)^{**} \\ 
   \multicolumn{5}{l}{\textbf{H4: (U/V)HNWIs}} \\ 
 & HNWIs (net worth $\geq$ 1M) & .06 \; (0.12) & -.07 \; (0.13) & -.05 \; (0.12) \\ 
   & HNWIs (net worth $\geq$ 5M) & .15 \; (0.11) & -.21 \; (0.11) & -.18 \; (0.10) \\ 
   & HNWIs (net worth $\geq$ 30M) & .04 \; (0.10) & -.06 \; (0.09) & -.04 \; (0.08) \\ 
   & HNWIs (net worth $\geq$ 200M) & -.10 \; (0.10) & -.13 \; (0.09) & -.08 \; (0.07) \\ 
   \multicolumn{5}{l}{\textbf{Controls}} \\ 
 & GDP per cap. & .42 \; (0.18)^{*} & .76 \; (0.20)^{***} & .84 \; (0.18)^{***} \\ 
   & modern/contemp. art museums (1990) & .09 \; (0.18) & .18 \; (0.32) & .22 \; (0.31) \\ 
   & modern/contemp. art museums (1990, squared) & -.05 \; (0.05) & -.05 \; (0.06) & -.05 \; (0.05) \\ 
   & Collectors in Artnews top 200 collector list & .09 \; (0.07) & .15 \; (0.09) & .11 \; (0.08) \\ 
   & PM density (country) & 1.21 \; (0.17)^{***} & .77 \; (0.24)^{**} & .66 \; (0.23)^{**} \\ 
   & PM density squared (country) & -.30 \; (0.08)^{***} & -.19 \; (0.05)^{***} & -.15 \; (0.04)^{***} \\ 
   & PM density (global) & -.44 \; (0.22)^{*} & -.04 \; (0.16) & -.00 \; (0.14) \\ 
   & PM density squared (global) & -.29 \; (0.15)^{*} & -.29 \; (0.12)^{*} & -.26 \; (0.12)^{*} \\ 
   & Nbr. PM closings (cumulative, global) & .15 \; (0.21) & -.12 \; (0.15) & -.09 \; (0.14) \\ 
   \hline
 & N & 1245 & 2514 & 3330 \\ 
   & No. countries & 85 & 150 & 161 \\ 
   & Log likelihood & -557.50 & -697.47 & -716.32 \\ 
   \hline 
 \multicolumn{4}{l}{\footnotesize{standard errors in parantheses.\textsuperscript{***}p \textless 0.001;\textsuperscript{**}p \textless 0.01;\textsuperscript{*}p \textless 0.05.}}
\end{tabular}
\caption{Negative binomial models of private museum founding rate} 
\label{tbl:regrslts_wcptblF}
\end{table}

Table \ref{tbl:regrslts_wcptblF} displays the regression results. 
As the estimation procedure involves the calculation of multiple models from different variable sets, table \ref{tbl:regrslts_wcptblF} displays for each variable and dataset the coefficient of the best-fitting model (coefficient variation due to different variable choice (e.g. the difference between the coefficients of cultural spending between a model with income inequality measured once via the Gini coefficient and once via 10\% concentration) generally does not result in qualitatively different coefficients; see figure \ref{fig:coef_violin} in the appendix). 
The log likelihoods similarly correspond to the best-fitting model. 
Coefficients can be interpreted as logged multipliers of a country's founding rate \parencite{Coxe_West_Aiken_2009_count}, therefore a coefficient of 0.25 corresponds to a \(e\)\textsuperscript{0.25} = 1.28 multiplier, i.e. a 28\% increase in average predicted country founding rate given a 1 standard deviation change in the independent variable (the independent variables are standardized), while a coefficient of -0.25 corresponds to a \(e\)\textsuperscript{-0.25} = 0.78 multiplier.



\paragraph*{Control Variables}





The coefficient of GDP per capita is positive and significant in all datasets. 
It is furthermore substantially larger in "DS --CuSp" (0.76) and "DS --CuSP/TMITR" (0.84) than in "DS all IVs" (0.42).
This between-dataset variation in coefficient size could indicate mediation of GDP through cultural spending (cf. Appendix \ref{app_mediation} and \ref{fig:oucoefchng}); however further investigation would be required to more clearly determine the mechanism at play. 





In all datasets, measures of density (legitimacy) and density squared (competition) correspond to the density-dependence paradigm, i.e. an inverted U-shaped relationship between density and private museum founding (indicated by the significant negative coefficient of the squared density measure on both country and global level for all datasets). 
Private museum founding thus is compatible with the density dependence paradigm, according to which potential founders are motivated by their peers' decision to establish a private museum up to the point where private museum numbers increased to the extent that additional founding does not appear advantageous anymore. 
However, while prospective founders might ultimately be disincentivized by growing private museum numbers, there is no evidence for a demotivating effect of  museum closures, as the coefficient of the cumulative number of museum closures is not significant and varies in sign between datasets.






Similary, the coefficients of the art field indicators (the number of museums of modern and contemporary art and of the number of collectors in the Artnews top collector ranking) are relatively small and insignificant in all datasets.





\paragraph*{Tax Incentives (H1)}




\begin{figure}[htbp]
\centering
\includegraphics[width=14cm]{figures/plt_v91_pred_taxinc.pdf}
\caption{\label{fig:pred_taxinc}Tax Incentives and Private Museum Founding: Adjusted Predictions at the Means (DS all IVs; population 100 mil.; 95\% CI)}
\end{figure}

The results for tax incentives largely support the hypothesis of tax incentives as a factor contributing to private museum founding:
In DS all IVs (as well as DS -CuSp), the interaction between tax deductibility of donations and top marginal income tax rates is positive and significant, which indicates that increases in top marginal income tax rates have a larger positive effect in countries where charitable donations are tax deductible than in countries where they are not.
Figure \ref{fig:pred_taxinc} visualizes how the impact of top marginal income tax rates depends on the presence of tax deductibility of donations.
In particular at higher top marginal income tax rates countries with tax deductibility of donations see more private museum foundings than countries where charitable donations are not tax deductible.
The fact that in the former countries collectors can receive substantial tax benefits appears to indeed incentivize potential founders to establish their museums.
A positive role of tax incentives is further supported by the positive and marginally significant (p = 0.10) coefficient of tax deductibility of donations in DS --CuSp/TMITR, indicating that countries with tax deductibility of donations see \(e\)\textsuperscript{0.72} = 2.1 times more private museum founding than countries without. 



However, it is worth noting that in DS all IVs and DS --CuSp at lower top marginal income tax rates countries without tax deductibility of charitable donations see more private museum foundings (figure \ref{fig:pred_taxinc}).
Under the assumption that founders are keen to reduce the net cost of their philanthropic activities, such results are unexpected as at these low tax rates tax deductibility of donations plays such a marginal role so that no substantial differences would be predicted between countries with and without tax deductibility of donations.  
Whether this finding is due to ommitted variable bias or a more non-linear relationship between tax rates and museum foundings remains the work for future research.
Despite such minor unexpected results, the combined findings point to tax incentives playing a substantive facilitative role in private museum founding. 



\paragraph*{Government Cultural Spending (H2)}

\begin{figure}[htbp]
\centering
\includegraphics[width=14cm]{figures/plt_v91_pred_smorc.pdf}
\caption{\label{fig:pred_smorc}Goverment Cultural Spending and Private Museum Founding: Adjusted Predictions at the Means (population 100 mil.; 95\% CI)}
\end{figure}

Government cultural spending is found to have a significant inverse-U shaped association with private museum founding.
While at lower values the association is positive, the effect decreases at higher levels and eventually turns negative (figure \ref{fig:pred_smorc}).
In particular, up to a value of -0.38/(2*-0.63) = 0.30 (corresponding to cultural spending of 439 USD per capita, approximately the 2020 government cultural spending of Slovakia (415 USD) or New Zealand (463 USD)) an increase in government spending corresponds to a higher predicted likelihood of private museum founding. 
This range of a positive association between government spending and private museum founding consists of 818 (65.7\%) of the country-years belonging to "DS all IVs"; it also contains 59.7\% of covered private museum foundings.
It is furthermore worth to consider that since "DS all IVs" consists disproportionately of developed countries, the true proportion of country-years in which higher government expenditure is associated with higher private museum is likely even higher.
Nevertheless, in the range above this turning point which (in 2020) includes countries prominent for their private museum populations such as Germany (660 USD), France (680 USD), the Netherlands (782 USD) or Switzerland (806 USD) increases in government spending are associated with lower rates of private museum founding (and conversely, decreases in government spending with higher rates of private museum founding).
The data thus supports both the crowding-out argument of the private museum literature which considers private museum founding  a consequence of declining government spending (H2) as well as the crowding-in argument raised by philanthropy scholars. 
However, little evidence is found for a lack of government spending facilitating private museum founding in countries  with emerging art scenes, as these are located in the range in which government spending is positively associated with private museum founding. 

\paragraph*{Inequality (H3)}



The association between inequality variables and private museum founding is mixed.
While income inequality is significantly associated with lower founding, wealth inequality variables are positively and significantly associated with the establishment of new private museums. 
A 1 SD increase of the income share of the top 10\% (8.9\% in 2020 in "DS all IVs", which corresponds approximately to the difference between Sweden (30.8\%) and Canada (39.7\%), or the within-country change in Lithuania from 2000 (32.5\%) to 2014 (41.1\%)) is associated with a decrease in private museum founding rate by a factor of \(e\)\textsuperscript{-0.47} = 0.63. 
Conversely, a 1 SD increase of the wealth share of the top 10\% (7.5\%; a difference in size similar to that between Denmark (50.2\%) and the Czech Republic (57.9\%) in 2020 in "DS all IVs", or to the longitudinal change in the United States from 1991 (64.1\%) to 2011 (71.7\%)) is associated with an increase in private museum founding rate by a factor of \(e\)\textsuperscript{0.33} = 1.40.
There is thus substantial evidence for H3 in regards of wealth inequality, but also strong empirical evidence against a positive association between income inequality and private museum founding. 


Two additional patterns are visible for both wealth and income inequality: 
Firstly, the effects are substantially larger in "DS --CuSP" and "DS --CuSp/TMITR".
Secondly, in all datasets the coefficients of the top percentile shares are larger (in absolute terms) than those of the gini coefficients.
Either one or both of these patterns possibly lead to the gini coefficients in "DS all IVs" to become insignificant; investigating the mechanisms behind these patterns constitutes fruitful approaches for future research. 



\paragraph*{HNWI (H4)}


Increases in the number of HNWI do not seem to be substantially associated with private museum founding as after controlling for other variables, the HNWI coefficients are firstly comparatively small relative to those of other indicators, secondly not statistically significant and thirdly occasionally negative, in particular in DS -CuSp and DS -CuSP/TMITR).
In these two datasets, the coefficient of the HNWI population with a wealth larger than 5 million USD in particular is substantially negative (-0.21 and -0.18) and marginally significant (p = 0.054 and p = 0.069 for DS -CuSp and DS -CuSP/TMITR, respectively).
Consequently, there is little evidence that HNWI rates are substantially positively associated with private museum founding, thus rejecting H4.



\subsubsection*{Counterfactual Analysis}

\begin{figure}[htbp]
\centering
\includegraphics[width=19cm]{figures/plt_v91_cntrfctl.pdf}
\caption{\label{fig:cntrfctl}Counterfactual simulations}
\end{figure}

Beyond interpreation of regression coefficients, the substantial association between the predictors and private museum founding can be illustrated by constructing a counterfactual.
In particular, I hold (separately for each variable) each country at the mean value of the years 2000 to 2004 for the entire time period, predict the number of openings for each country year under this counterfactual scenario using the original coefficients, sum up the predictions for all country years and compare with the observed number of openings.
Results are show in figure \ref{fig:cntrfctl} (the longitudinal changes of the predictor variables are shown in appendix \ref{app_velp}, \ref{fig:velp}).
Positive values for a variable (e.g. GDP) indicate that in the actual world, more private museums were founded than would have been had the variable stayed at its 2000-2004 level; the change of the variable can thus be understood as having led to an increase in private museum founding.
In the case of GDP, it can be seen that changes since 2000-2004 are, depending on dataset, responsible for 18 to 75 additional private museums foundings.
Conversely, negative values (e.g. for top marginal income tax rates) indicate that by changing in the real world, less private museums are founded than predicted in a world where variables stay at 2000-2004 levels.
For top marginal tax rates it can thus be seen that if they had stayed at their mean level of 2000-2004, 18 to 40 additional private museums would have been founded (in DS all IVs and DS -CuSp, respectively).
Despite only capturing the within-country variation, this analysis shows that the predictors are substantially associated with private museum founding.


\subsection*{Discussion and conclusion}

As the first quantitative analysis of private museum proliferation, this study has critically examined several factors that have been suggested as explanations for the rise of private museums in recent decades.
To some extent the findings are in line with the literature, particularly in the case of tax incentives, which are found to positively contribute to private museum founding.
In other domains, this research complicates existing arguments, such as in the case of government spending, which shows an inverse-U shaped relationship (and thus corresponds to crowding-in at lower levels, and crowding out at higher levels), and inequality, where wealth disparities are found to be positively associated with private museum founding, but income inequalities appear to have a negative effect. 
In yet another area, the increase in numbers of the very wealthy, no substantial relation to private museum founding has been found. 
This study thus substantially advances the understanding of a major art world development of the recent decades.




Despite the combination of multiple data sources, an innovative modeling strategy sensitive to variations in the time predictors take to act on the dependent variable as well as robust measurement of several key concepts via different operationalizations this study is limited by a number of factors. 
As interest in the organizational form of private museums has risen only recently, data collection has been both retrospective (which might be less inclusive than continuous measurements based on for example industry directories) and primarily based on English-language media.
Furthermore, multiple predictors are exhibiting substantial limitations in measuring their intended concepts, such as the measure of tax deductibility of donations and non-private art museum populations variables (which are only cross-sectional and in the former case not art-sector specific), the WID HNWI data (making HNWI estimation beyond 200 million USD infeasible) and the Artnews top 200 collector ranking data (which does not capture absolute changes of collector populations).
Additionally, substantial gaps in data coverage of cultural spending and top marginal income tax rates have so far necessitated the use of multiple datasets.
While this approach was of substantial value, as it allowed to investigate variables with varying coverage and  allowed to include many countries which had not received much attention by the private museum literature, better coverage of some variables would nevertheless provide even further insights.
Finally, further relationships between variables (such as mediation or moderation) as well as nonlinear relationships to private museum foundings beyond those considered might be present which, if not taken into account, might bias coefficient estimates.
Mediation in particular might be present for density estimates, as the effect of structural factors might be mediated through the private museum population.
Further disentangling these effects, improving measurement precision and elaborating the underlying theoretical mechanisms thus constitute promising approaches for future research. 

\begin{sloppypar}
\printbibliography
\end{sloppypar}

\clearpage
\FloatBarrier


\section*{Methodological Appendix}


\appendix
\setcounter{page}{1}


\counterwithin{table}{subsection}
\counterwithin{figure}{subsection}
\renewcommand{\thesubsection}{\Alph{subsection}}



\subsection{Data Coverage}
\label{app_data_coverage}
\begin{figure}[htbp]
\centering
\includegraphics[width=18cm]{figures/plt_v91_vrbl_cycnt.pdf}
\caption{\label{fig:vrbl_cycnt}Number of countries with per year per variable}
\end{figure}



\begin{figure}[htbp]
\centering
\includegraphics[width=18cm]{figures/plt_v91_cbn_cycnt.pdf}
\caption{\label{fig:cbn_cycnt}Number of countries per year per variable combination}
\end{figure}



Figure \ref{fig:vrbl_cycnt} shows the country-year coverage of the main variables (other HNWI thresholds and inequality shares follow the ones depicted).
In particular it can be seen that the coverage of wealth variables in the WID improves substantially from 1995 onwards, a pattern that is to a lesser extent also visible in indicators of cultural spending and top marginal income tax rates.
Given this state of data coverage and the fact that five years of subsequent data are required for lag length optimization, the observation period for the largest proportion of countries starts in the year 2000 or later  (figure \ref{fig:cbn_cycnt}, the start 1995 is set by the availability of Artnews collector ranking data from 1990 onwards).

\subsection{Combination composition}


% latex table generated in R 4.2.3 by xtable 1.8-4 package
% Fri May  5 12:09:40 2023
\begin{table}[ht]
\centering
\begin{tabular}{p{2.5cm}rrrrrr}
  \hline 
 & \multicolumn{2}{c}{DS all IVs} & \multicolumn{2}{c}{DS --CuSp} & \multicolumn{2}{c}{DS --CuSp/TMITR} \\ 
\cmidrule(r){2-3}\cmidrule(r){4-5}\cmidrule(r){6-7} 
 region & N & Percent & N & Percent & N & Percent \\ 
  \hline
Africa & 108 &  8.6\% & 622 & 24.3\% & 1 088 & 32.1\% \\ 
  Asia & 334 & 26.7\% & 653 & 25.5\% & 938 & 27.6\% \\ 
  Europe & 663 & 53.0\% & 715 & 27.9\% & 779 & 23.0\% \\ 
  Latin America & 75 &  6.0\% & 462 & 18.0\% & 478 & 14.1\% \\ 
  North America & 34 &  2.7\% & 47 &  1.8\% & 47 &  1.4\% \\ 
  Oceania & 36 &  2.9\% & 63 &  2.5\% & 63 &  1.9\% \\ 
   \hline
\end{tabular}
\caption{Dataset composition by region} 
\label{tbl:cbn_cpsgn}
\end{table}


Table \ref{tbl:cbn_cpsgn} shows the coverage of datasets by region. 
Relative between-dataset differences are particularly strong for Africa, Latin America and Europe.
While Europe with \textbf{more than half of the country-years} is "DS all IVs" constitutes the largest region (indicating that European countries report relatively detailed statistics on government spending), its share declines to \textbf{XX\%} and \textbf{YY\%} in "DS -CuSp" and "DS -CuSp/TMITR", respectively.
Conversely, Africa and Latin America claim larger shares in the larger datasets, with the former becoming the region contributing the largest nubmer of country years (\textbf{X} or \textbf{Y\%}) in "DS -CuSp/TMITR".
The proportion of North America and Oceania also declines in larger datasets, but this results in comparatively less changes in dataset composition as these regions consist of relatively few countries. 



with the shares of the former increasing with dataset size in country-years (\textbf{list percentages here}, while she latter 





\subsection{Data processing}
\label{app_data_processing}
\subsubsection{Cultural spending data source combination}


This combination of multiple data sources requires the harmonization of different reporting standards: 
Whereas the the IMF and Eurostat report data exclusively as "Total government expenditure" (TLYCG), the UN uses "Final consumption expenditure" (P3CG); the OECD reports data in both formats ("Total government expenditure" is calculated from "Final consumption expenditure" as well as a number of other items, such as compensation of employees and subsidies). 
Moreoever, within each format minor variations exist between data sources, the data for a country-year is thus chosen in order of OECD followed by UN for P3CG, and OECD followed by IMF followed by Eurostat for TLYCG.
As the overall goal is to create a complete picture of government expenditure, Total government expenditure is estimated from Final consumption expenditure for country-years where data is only available for the latter.
For countries where TLYCG and P3CG series have some overlap and years exist with P3CG data but not TLYCG data, a country-specific scaler to convert P3CG to TLYCG is constructed from overlapping years, which is then used to impute TLYCG for the years in which only P3CG data is available.
For countries with only P3CG data, the average ratio of all country-years with both P3CG and TLYCG data is chosen to impute TLYCG (as on average P3CG is 58\% of TLYCG, the average scaler is 1/0.58 = 1.72). 
Amounts are reported in current local currency units and converted to 2021 USD using price indices and purchasing power parity adjusted exchange rates from the World Inequality Database (WID,  \citeyear{WID_2021_WID}).
I estimate coefficients of the cultural spending variable as well as its squared term to account for potential non-linearities similar to those present in the density dependence paradigm (\cite{Hannan_1992_dynamics};  elaborated in section Control Variables); in particular, the crowding-out argument appears to indicate competition, while the crowding-in argument bears similarities to legitimation. 


\subsubsection{Imputation}


Due to the exploratory approach of testing variables at lag lengths varying from one to five years, missing values can potentialily substantially limit the number of country years as a single missing value leads to the exclusion of the next five years.
To avoid such loss of data, missing values in the country year time series which are parts of gaps of up to three years are linearly imputed.
This primarily concerns government cultural spending (25 country years imputed), and to a lesser extent wealth inequality measures, HNWI measures and population size (7, 4 and 3 country years imputed, respectively).


Furthermore, it was not possible to find the exact closing years for 25 private museums which were found to be no longer open.
These cases constitute a challenge for calcuating accurate density measures: 
Leaving out these museums entirely would lead to underestimated density estimates, while treating these museums as remaining open would overestimate private museum density as they were observed to be no longer open. 
Either method can substantially bias density estimates as in particular in countries with only a few private museums, a private museum more or less can have large impacts on per capita private museum rates.
To be able to still use these cases in density estimates, closing year was imputed via linear regression based on the relationship between number of years opened and closing year of the museums for which both were available (n=53, R\textsuperscript{2} = 0.68).
While imputed closing years are likely not always accurate, the resulting density estimates are likely more accurate than they would have been if closed museums had been excluded completely or treated as having remained open, especially given the high R\textsuperscript{2} of imputation regression used for the traning data.




\subsection{Coefficient Distribution}



\begin{figure}[htbp]
\centering
\includegraphics[width=18cm]{figures/plt_v91_coef_violin.pdf}
\caption{\label{fig:coef_violin}Distribution of coefficient point estimates (Gaussian kernel density estimate; bandwidth = 0.04)}
\end{figure}



\subsection{Model improvement given inclusion of variables}

\begin{figure}[htbp]
\centering
\includegraphics[width=18cm]{figures/plt_v91_oneout_llrt_lldiff.pdf}
\caption{\label{fig:oneout_llrt_lldiff}Model improvement given variable inclusion (Gaussian kernel density estimate; bandwidth = 0.4)}
\end{figure}


\begin{figure}[htbp]
\centering
\includegraphics[width=18cm]{figures/plt_v91_oneout_llrt_z.pdf}
\caption{\label{fig:oneout_llrt_z}Distribution of Z-score of log-likelihood ratio test p-value (Gaussian kernel density estimate; bandwidth = 0.1)}
\end{figure}

To investigate whether a variable improves the model, a comparison is made between the full model and the full model without the variable in question.
For each dataset there are 36 models (due to variables choices for HWNI (4 different thresholds) and inequality measures (1 of 3 for both wealth and income inequality)), resulting in 108 models in total. 
For each variable in each of these models a reduced model is constructed by removing the variable in question and comparing model fit to the full model.
If a variable has a squared term or interaction, it is removed together with the main term.
Furthermore additional reduced models is constructed, namely one without the four density variables (country and global density linear and squared), as well as one without the density variables and closings.
Given that the datasets differ in their number of variables, a different number of reduced models is calculated per dataset, in particular 684 for "DS all IVs", 612 for "DS --CuSp", and 540 for "DS --CuSp/TMITR". 
The lags of the reduced models are not optimized due to computational limitations. 



Figure \ref{fig:oneout_llrt_lldiff} shows the distribution of differences in log-likelihood between the full and reduced models per variable and dataset.
Furthermore, a likelihood ratio test (\(LR = 2[LL_{reduced} - LL_{full}]\)) is conducted to compare each reduced to its corresponding full model.
The likelihood ratio statistic follows a Chi-square distribution; its corresponding p-value was converted to a z-score to facilitate interpretation.
The distribution of z-scores per variable and dataset is shown in figure \ref{fig:oneout_llrt_z}.
Both analysis correspond in large parts to the results of the main regression analysis  insofar as variables with significant coefficient correspond to significant and/or substantial model improvements.
There are however a few exceptions, such as tax deductibility of donations in "DS -CuSp/TMITR", GDP per capita in "DS all IVs" as well as some  wealth inequality variables in "DS all IVs" in which a significant coefficient does not always correspond to a significant model improvement. 



\subsection{Influence of variable inclusion on regression coefficients}
\label{app_mediation}
\begin{landscape}

\begin{figure}[htbp]
\centering
\includegraphics[width=24cm]{figures/plt_v91_oucoefchng.pdf}
\caption{\label{fig:oucoefchng}Coefficient changes given addition of other variables}
\end{figure}

\end{landscape}

I furthermore analyze the coefficients of the restricted models to investigate potential mediation; results are presented in figure \ref{fig:oucoefchng}.
The variables (or variable sets of all density variables and all density variables plus closures) that are added are placed on the x-axis, the coefficients of the full model are placed on the y-axis;
Each point shows the average difference between the coefficient of the full and the restricted model and can be understood as the effect that adding variable v\textsubscript{x} to the model has on the coefficient of variable v\textsubscript{y}.
For example, if GDP has a coefficient of 0.3 in the full model and one of 0.1 in the restricted model (e.g. one without cultural spending), the difference is 0.3 - 0.1 = 0.2; thus adding cultural spending to the model results in an increase of the GDP coefficient by 0.2. 
Positive coefficient changes (i.e. a larger coefficients in the full model than in the restricted model) are colored as red, negative coefficient changes as blue; points are furthermore scaled by the absolute coefficient size to compare both positive and negative changes. 


A number of findings can be gleaned from this analysis.
Firstly, wealth and income inequality appear "mutually reinforcing".
The inclusion of income inequality increases the coefficients of wealth inequality (which is positive in the full model) and the inclusion of wealth inequality decreases further the negative coefficient of income inequality (which in the full model is negative).
This unexpected pattern (as well as the divergent inequality in general) clearly calls for further research to disentangle relations of inequality.


Secondly, a number of variables appear to partly mediate GDP.
The coefficient of GDP decreases as other variables are added, which indicates that part of the effect is mediated through these variables.
This in particular concerns the effects of density, tax incentives and cultural spending, and to a lesser extent the effect of inequalities (for "DS --CuSp" and "DS --CuSp/TMITR") and some HNWI measures (for "DS all IVs").
Conversly, adding GDP to a model in which it was not included before reduces the coefficients of HNWIs and museums of modern/contemporary art.


Finally, there are a number of substantial coefficient changes given inclusion which do not offer an easy explanation, such as decreases of the coefficient of modern and contemporary art museums when including income inequality measures, increases of the linear density term when adding wealth inequality variables, increases of income inequality coefficients when adding density measures, as well as changes to the intercept in both directions when adding different variables. 
Exploring the various cases of potential mediation (or other forms of variable relations) is beyond the scope of the paper, but constitutes a promising start for future research.






\subsection{Lag Choice}

\begin{figure}[htbp]
\centering
\includegraphics[width=18cm]{figures/plt_v91_lag_dens.pdf}
\caption{\label{fig:lag_dens}Lag choice distribution}
\end{figure}

Figure \ref{fig:lag_dens} shows the distribution of the lag of the coefficient after optimization.
As often time lags different from one year are chosen (which would likely constitute the default if they were not varied), it can be seen that allowing the lag to vary substantially increases model fit. 
It furthermore seems to be the case that the HNWI coefficients (which are not significant) vary the most in regards to their lag choice (which is plausible since a non-substantial overall effect could imply that the particular lag does not matter much). 




\subsection{Multicollinearity}


\begin{figure}[htbp]
\centering
\includegraphics[width=18cm]{figures/plt_v91_vif.pdf}
\caption{\label{fig:vif}Distribution of VIF estimates (Gaussian kernel density estimate; bandwidth = 0.1)}
\end{figure}


VIFs were calculated for the best-fitting model of each variable set and dataset (108 models in total given 1 of 4 HWNI variables \texttimes{} 1 of 3 income inequality variables \texttimes{} 1 of 3 wealth inequality variables \texttimes{} 1 of 3 datasets) using the R \texttt{performance} package \parencite{Luedecke_etal_2021_performance}. 
As squared variables and interactions can result in high VIFs without substantial collinearity, I calculate VIFs once for the full model and once after excluding squared variables and interactions.
Figure \ref{fig:vif} shows the distribution of the variance inflation factors.
While VIFs can be substantial when including squared variables and interactions, no multicollinearity issues are present when focusing only on the linear variables (all VIFs < 10, all VIFs except global density (which after removing squared variables is still based on the same data as global density) < 5).



\subsection{Longitudinal development}
\label{app_velp}
\begin{landscape}

\begin{figure}[htbp]
\centering
\includegraphics[width=24cm]{figures/plt_v91_velp.pdf}
\caption{\label{fig:velp}Results of regressing longitudinal variables on year}
\end{figure}

\end{landscape}

The within-country changes were analyzed to characterize the development of the longitudinal variables over the observation period.
In particular, for each variable a separate regression model was run which regresses the variable in question at lag 0 against year while allowing slopes and intercepts to vary by country (year is set to 0 in 1995, the beginning of the observation period).
Results are presented in figure \ref{fig:velp}.
The histogram shows the distribution of country slopes, while the dot and whiskper shows the overall slope estimate with a 95\% confidence interval.
For example the overall slope of top marginal income tax rates is -0.010 (indicating an average yearly decrease of top marginal income tax rates by 0.13 percentage points), however the histogram shows that countries can substantially diverge from this overall slope: 
Slopes of countries (with at least 20 years of data) can range from a minimum of -0.117 (a yearly decrease by 1.6 percentage points in the case of Hungary) to a maximum of 0.063 (a yearly increase of 0.9 percentage points in the case of Portugal), with 25\% and 75\% slope quantiles corresponding to -0.025 (-0.34\%) and 0.007 (0.10\%), respectively.
Furthermore, the correlation between slope and intercept is included:
A positive correlation indicates that countries with higher constants (i.e. predicted value in 1995) achieve higher growth over the observation period than countries with lower constants (thereby "extending their lead"), whereas negative correlations indicate that countries with lower constants experience higher growth (thereby "catching up").
\end{document}