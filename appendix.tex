% Created 2023-07-06 do 14:13
% Intended LaTeX compiler: pdflatex
\documentclass[11pt]{article}

\usepackage[hyphens]{url}                
\usepackage{hyperref}
\usepackage[hyphenbreaks]{breakurl}
\usepackage{rotating}
\usepackage{wrapfig}
\usepackage{pdflscape}
\usepackage{fixltx2e}
\usepackage{graphicx}
\usepackage{amsmath}
\usepackage{amsfonts}
\usepackage[section]{placeins}
\usepackage{dirtree}
\usepackage{siunitx}
\usepackage{afterpage}
\usepackage{pdflscape}
\usepackage{svg}
\usepackage[export]{adjustbox}


\usepackage{booktabs}
\usepackage{dcolumn}
\makeatletter
\newcolumntype{D}[3]{>{\textfont0=\the\font\DC@{#1}{#2}{#3}}c<{\DC@end}}
\makeatother


\newcolumntype{L}{>{$}l<{$}}

\usepackage{bibentry}

\sisetup{detect-all}

\sloppy
\usepackage{scalerel,stackengine}

\stackMath
\newcommand\reallywidehat[1]{%
\savestack{\tmpbox}{\stretchto{%
  \scaleto{%
    \scalerel*[\widthof{\ensuremath{#1}}]{\kern-.6pt\bigwedge\kern-.6pt}%
    {\rule[-\textheight/2]{1ex}{\textheight}}%WIDTH-LIMITED BIG WEDGE
  }{\textheight}% 
}{0.5ex}}%
\stackon[1pt]{#1}{\tmpbox}%
}

\usepackage{caption}
\usepackage[draft]{todonotes}

\captionsetup{skip=0pt}
\usepackage[utf8]{inputenc}
\usepackage[style=apa, backend=biber]{biblatex} 
\usepackage[english, american]{babel}
\DeclareLanguageMapping{american}{american-apa}
\DeclareFieldFormat{apacase}{#1}

\usepackage[T1]{fontenc}
\usepackage{csquotes}

\addbibresource{/home/johannes/Dropbox/references.bib}
\addbibresource{/home/johannes/Dropbox/references2.bib}

\usepackage{floatrow}

\usepackage{listings}
\usepackage{xcolor}
\usepackage{colortbl}

\lstset{
  language=R,                    
  basicstyle=\footnotesize,      
  numbers=left,                  
  numberstyle=\tiny\color{gray}, 
  stepnumber=1,                  
  numbersep=5pt,                 
  backgroundcolor=\color{white}, 
  showspaces=false,              
  showstringspaces=false,        
  showtabs=false,                
  frame=single,                  
  rulecolor=\color{black},       
  tabsize=2,                     
  captionpos=b,                  
  breaklines=true,               
  breakatwhitespace=false,       
  title=\lstname,                
  keywordstyle=\color{red},     
  commentstyle=\color{blue},  
  stringstyle=\color{violet},     
  escapeinside={\%*}{*)},        
  morekeywords={*,...}           
} 




\usepackage{crimson}
\usepackage{microtype}


\usepackage{fancyhdr}
\usepackage{setspace}
\singlespace
\usepackage{longtable}
\usepackage{subfig}
\usepackage[a4paper, total={18cm, 24cm}]{geometry}

\pagestyle{fancy}
\fancyhf{}
\renewcommand{\headrulewidth}{0pt}
\renewcommand{\maketitle}{}

\usepackage{enumitem}
\setlist[itemize]{topsep=0pt,itemsep=0pt,parsep=0pt,partopsep=0pt}

\usepackage{multicol}
\setlength\multicolsep{0pt}

\usepackage{array}
\usepackage{caption}
\usepackage{graphicx}
\usepackage{siunitx}
\usepackage[normalem]{ulem}
\usepackage{colortbl}
\usepackage{multirow}
\usepackage{hhline}
\usepackage{calc}
\usepackage{tabularx}
\usepackage{threeparttable}
\usepackage{wrapfig}
\usepackage{adjustbox}
\usepackage{hyperref}



\newlist{propertyList}{itemize}{1}
\setlist[propertyList]{
  label=\textbullet,
  noitemsep,
  leftmargin=10pt,
  before=\begin{multicols}{3},
  after=\end{multicols}
  }

\cfoot {Johannes Aengenheyster}
\rfoot {\thepage}

\listfiles

\setlength{\parindent}{1.2cm}
\author{Johannes Aengenheyster}
\date{\today}
\title{}
\hypersetup{
 pdfauthor={Johannes Aengenheyster},
 pdftitle={},
 pdfkeywords={},
 pdfsubject={},
 pdfcreator={Emacs 28.2 (Org mode 9.6.6)}, 
 pdflang={English}}
\begin{document}


\section*{Appendix}


\subsection*{Data Coverage}

\begin{figure}[htbp]
\centering
\includegraphics[width=18cm]{figures/plt_v91_vrbl_cycnt.pdf}
\caption{\label{fig:vrbl_cycnt}Number of countries with per year per variable}
\end{figure}



\begin{figure}[htbp]
\centering
\includegraphics[width=18cm]{figures/plt_v91_cbn_cycnt.pdf}
\caption{\label{fig:cbn_cycnt}Number of countries per year per variable combination}
\end{figure}



Figure \ref{fig:vrbl_cycnt} shows the country-year coverage of the main variables (other HNWI thresholds and inequality shares follow the ones depicted).
In particular it can be seen that the coverage of wealth variables in the WID improves substantially from 1995 onwards, a pattern that is to a lesser extent also visible in indicators of cultural spending and top marginal income tax rates.
Given this state of data coverage and the fact that five years of subsequent data are required for lag length optimization, the observation period for the largest proportion of countries starts in the year 2000 or later  (figure \ref{fig:cbn_cycnt}, the start 1995 is set by the availability of Artnews collector ranking data from 1990 onwards).

\subsection*{Combination composition}


% latex table generated in R 4.2.3 by xtable 1.8-4 package
% Fri May  5 12:09:40 2023
\begin{table}[ht]
\centering
\begin{tabular}{p{2.5cm}rrrrrr}
  \hline 
 & \multicolumn{2}{c}{DS all IVs} & \multicolumn{2}{c}{DS --CuSp} & \multicolumn{2}{c}{DS --CuSp/TMITR} \\ 
\cmidrule(r){2-3}\cmidrule(r){4-5}\cmidrule(r){6-7} 
 region & N & Percent & N & Percent & N & Percent \\ 
  \hline
Africa & 108 &  8.6\% & 622 & 24.3\% & 1 088 & 32.1\% \\ 
  Asia & 334 & 26.7\% & 653 & 25.5\% & 938 & 27.6\% \\ 
  Europe & 663 & 53.0\% & 715 & 27.9\% & 779 & 23.0\% \\ 
  Latin America & 75 &  6.0\% & 462 & 18.0\% & 478 & 14.1\% \\ 
  North America & 34 &  2.7\% & 47 &  1.8\% & 47 &  1.4\% \\ 
  Oceania & 36 &  2.9\% & 63 &  2.5\% & 63 &  1.9\% \\ 
   \hline
\end{tabular}
\caption{Dataset composition by region} 
\label{tbl:cbn_cpsgn}
\end{table}


Table \ref{tbl:cbn_cpsgn} shows the coverage of datasets by region. 
Relative between-dataset differences are particularly strong for Africa, Latin America and Europe.
While Europe with \textbf{more than half of the country-years} is "DS all IVs" constitutes the largest region (indicating that European countries report relatively detailed statistics on government spending), its share declines to \textbf{XX\%} and \textbf{YY\%} in "DS -CuSp" and "DS -CuSp/TMITR", respectively.
Conversely, Africa and Latin America claim larger shares in the larger datasets, with the former becoming the region contributing the largest nubmer of country years (\textbf{X} or \textbf{Y\%}) in "DS -CuSp/TMITR".
The proportion of North America and Oceania also declines in larger datasets, but this results in comparatively less changes in dataset composition as these regions consist of relatively few countries. 



with the shares of the former increasing with dataset size in country-years (\textbf{list percentages here}, while she latter 





\subsection*{Data processing}


\subsubsection*{Cultural spending data source combination}


This combination of multiple data sources requires the harmonization of different reporting standards: 
Whereas the the IMF and Eurostat report data exclusively as "Total government expenditure" (TLYCG), the UN uses "Final consumption expenditure" (P3CG); the OECD reports data in both formats ("Total government expenditure" is calculated from "Final consumption expenditure" as well as a number of other items, such as compensation of employees and subsidies). 
Moreoever, within each format minor variations exist between data sources, the data for a country-year is thus chosen in order of OECD followed by UN for P3CG, and OECD followed by IMF followed by Eurostat for TLYCG.
As the overall goal is to create a complete picture of government expenditure, Total government expenditure is estimated from Final consumption expenditure for country-years where data is only available for the latter.
For countries where TLYCG and P3CG series have some overlap and years exist with P3CG data but not TLYCG data, a country-specific scaler to convert P3CG to TLYCG is constructed from overlapping years, which is then used to impute TLYCG for the years in which only P3CG data is available.
For countries with only P3CG data, the average ratio of all country-years with both P3CG and TLYCG data is chosen to impute TLYCG (as on average P3CG is 58\% of TLYCG, the average scaler is 1/0.58 = 1.72). 
Amounts are reported in current local currency units and converted to 2021 USD using price indices and purchasing power parity adjusted exchange rates from the World Inequality Database (WID,  \citeyear{WID_2021_WID}).
I estimate coefficients of the cultural spending variable as well as its squared term to account for potential non-linearities similar to those present in the density dependence paradigm (\cite{Hannan_1992_dynamics};  elaborated in section Control Variables); in particular, the crowding-out argument appears to indicate competition, while the crowding-in argument bears similarities to legitimation. 


\subsubsection*{Imputation}


Due to the exploratory approach of testing variables at lag lengths varying from one to five years, missing values can potentialily substantially limit the number of country years as a single missing value leads to the exclusion of the next five years.
To avoid such loss of data, missing values in the country year time series which are parts of gaps of up to three years are linearly imputed.
This primarily concerns government cultural spending (25 country years imputed), and to a lesser extent wealth inequality measures, HNWI measures and population size (7, 4 and 3 country years imputed, respectively).


Furthermore, it was not possible to find the exact closing years for 25 private museums which were found to be no longer open.
These cases constitute a challenge for calcuating accurate density measures: 
Leaving out these museums entirely would lead to underestimated density estimates, while treating these museums as remaining open would overestimate private museum density as they were observed to be no longer open. 
Either method can substantially bias density estimates as in particular in countries with only a few private museums, a private museum more or less can have large impacts on per capita private museum rates.
To be able to still use these cases in density estimates, closing year was imputed via linear regression based on the relationship between number of years opened and closing year of the museums for which both were available (n=53, R\textsuperscript{2} = 0.68).
While imputed closing years are likely not always accurate, the resulting density estimates are likely more accurate than they would have been if closed museums had been excluded completely or treated as having remained open, especially given the high R\textsuperscript{2} of imputation regression used for the traning data.




\subsection*{Coefficient Distribution}


\begin{figure}[htbp]
\centering
\includegraphics[width=18cm]{figures/plt_v91_coef_violin.pdf}
\caption{\label{fig:coef_violin}Distribution of coefficient point estimates (Gaussian kernel density estimate; bandwidth = 0.04)}
\end{figure}



\subsection*{Model improvement given inclusion of variables}

\begin{figure}[htbp]
\centering
\includegraphics[width=18cm]{figures/plt_v91_oneout_llrt_lldiff.pdf}
\caption{\label{fig:oneout_llrt_lldiff}Model improvement given variable inclusion (Gaussian kernel density estimate; bandwidth = 0.4)}
\end{figure}


\begin{figure}[htbp]
\centering
\includegraphics[width=18cm]{figures/plt_v91_oneout_llrt_z.pdf}
\caption{\label{fig:oneout_llrt_z}Distribution of Z-score of log-likelihood ratio test p-value (Gaussian kernel density estimate; bandwidth = 0.1)}
\end{figure}

To investigate whether a variable improves the model, a comparison is made between the full model and the full model without the variable in question.
For each dataset there are 36 models (due to variables choices for HWNI (4 different thresholds) and inequality measures (1 of 3 for both wealth and income inequality)), resulting in 108 models in total. 
For each variable in each of these models a reduced model is constructed by removing the variable in question and comparing model fit to the full model.
If a variable had a squared term or interaction, it was removed together with the main term.
Furthermore additional reduced models were constructed, namely one without the four density variables (country and global density linear and squared), as well as one without the density variables and closings.
Given that the datasets differ in their number of variables, a different number of reduced models is calculated per dataset, in particular \textbf{684} for "DS all IVs", \textbf{612} for "DS --CuSp", and \textbf{540} for "DS --CuSp/TMITR". 
The lags of the reduced models were not optimized due to computational limitations. 



Figure \ref{fig:oneout_llrt_lldiff} shows the distribution of differences in log-likelihood between the full and reduced models per variable and dataset.
Furthermore, a likelihood ratio test (\(LR = 2[LL_{reduced} - LL_{full}]\)) was conducted to compare each reduced to its corresponding full model.
The likelihood ratio statistic follows a Chi-square distribution; its corresponding p-value was converted to a z-score to facilitate interpretation.
The distribution of z-scores per variable and dataset is shown in figure \ref{fig:oneout_llrt_z}.


Both analysis correspond in large parts to the results of the main regression analysis \textbf{table X} insofar as variables with significant coefficient correspond to significant and/or substantial model improvements.
There are however a few exceptions, such as tax deductibility of donations in "DS -CuSp/TMITR", GDP per capita in "DS all IVs" as well as some  wealth inequality variables in "DS all IVs" in which a significant coefficient does not always correspond to a significant model improvement. 



\subsection*{Mediation: unsure if to use at all}


\begin{landscape}

\begin{figure}[htbp]
\centering
\includegraphics[width=24cm]{figures/plt_v91_oucoefchng.pdf}
\caption{\label{fig:oucoefchng}Coefficient changes given addition of other variables}
\end{figure}

\end{landscape}

I furthermore analyze the coefficients of the restricted models to investigate potential mediation; results are presented in figure \ref{fig:oucoefchng}.
The variables (or variable sets of all density variables and all density variables plus closures) that are added are placed on the x-axis, the coefficients of the full model are placed on the y-axis;
Each point shows the average difference between the coefficient of the full and the restricted model and can be understood as the effect that adding variable v\textsubscript{x} to the model has on the coefficient of variable v\textsubscript{y}.
For example, if GDP has a coefficient of 0.3 in the full model and one of 0.1 in the restricted model (e.g. one without cultural spending), the difference is 0.3 - 0.1 = 0.2; thus adding cultural spending to the model results in an increase of the GDP coefficient by 0.2. 
Positive coefficient changes (i.e. a larger coefficients in the full model than in the restricted model) are colored as red, negative coefficient changes as blue; points are furthermore scaled by the absolute coefficient size to compare both positive and negative changes. 


A number of findings can be gleaned from this analysis.
Firstly, wealth and income inequality appear "mutually reinforcing".
The inclusion of income inequality increases the coefficients of wealth inequality (which is positive in the full model) and the inclusion of wealth inequality decreases further the negative coefficient of income inequality (which in the full model is negative).
This unexpected pattern (as well as the divergent inequality in general) clearly calls for further research to disentangle relations of inequality.


Secondly, a number of variables appear to partly mediate GDP.
The coefficient of GDP decreases as other variables are added, which indicates that part of the effect is mediated through these variables.
This in particular concerns the effects of density, tax incentives and cultural spending, and to a lesser extent the effect of inequalities (for "DS --CuSp" and "DS --CuSp/TMITR") and some HNWI measures (for "DS all IVs").
Conversly, adding GDP to a model in which it was not included before reduces the coefficients of HNWIs, museums of modern/contemporary art and country-level density for "DS --CuSp" and "DS --CuSp/TMITR". 
does it need to be mediation? could be any kind of covariance? 






\subsection*{lag choice}

\begin{figure}[htbp]
\centering
\includegraphics[width=18cm]{figures/plt_v91_lag_dens.pdf}
\caption{\label{fig:lag_dens}Distribution of lag choice after optimization}
\end{figure}

Figure \ref{fig:lag_dens} shows the distribution of the lag of the coefficient after optimization.
As often time lags different from one year are chosen (which would likely constitute the default if they were not varied), it can be seen that allowing the lag to vary substantially increases model fit. 
It furthermore seem to be the case that the HNWI coefficients (which are not significant) vary the most in regards to their lag choice (which is plausible since a non-substantial overall effect could imply that the particular lag does not matter much). 




\subsection*{Multicollinearity}


\begin{figure}[htbp]
\centering
\includegraphics[width=18cm]{figures/plt_v91_vif.pdf}
\caption{\label{fig:vif}Distribution of VIF estimates (Gaussian kernel density estimate; bandwidth = 0.1)}
\end{figure}


VIFs were calculated for the best-fitting model of each variable set and dataset (108 models in total given 1 of 4 HWNI variables \texttimes{} 1 of 3 income inequality variables \texttimes{} 1 of 3 wealth inequality variables \texttimes{} 1 of 3 datasets) using the R \texttt{performance} package \parencite{Luedecke_etal_2021_performance}. 
As squared variables and interactions can result in high VIFs without substantial collinearity, I calculate VIFs once for the full model and once after excluding squared variables and interactions.
Figure \ref{fig:vif} shows the distribution of the variance inflation factors.
While VIFs can be substantial when including squared variables and interactions, no multicollinearity issues are present when focusing only on the linear variables (all VIFs < 10, all VIFs except global density (which after removing squared variables is still based on the same data as global density) < 5).



\subsection*{Longitudinal development}


\begin{landscape}

\begin{figure}[htbp]
\centering
\includegraphics[width=24cm]{figures/plt_v91_velp.pdf}
\caption{\label{fig:velp}Results of regressing longitudinal variables on year}
\end{figure}

\end{landscape}

Next to these overall statistics, the within-country changes were analyzed to characterize the development of the longitudinal variables over the observation period.
In particular, for each variable a separate regression model was run which regresses the variable in question at lag 0 against year while allowing slopes and intercepts to vary by country (year was 0 in 1995, the beginning of the observation period).
\textbf{can only cover linear country trends, especially since changes are usually reported as percentage change}
Results are presented in figure \ref{fig:velp}.
The histogram shows the distribution of country slopes, while the dot and whiskper shows the overall slope estimate with a 95\% confidence interval.
For example the overall slope of top marginal income tax rates is -0.010 (indicating an average yearly decrease of top marginal income tax rates by 0.13 percentage points), however the histogram shows that countries can substantially diverge from this overall slope: 
Slopes of countries (with at least 20 years of data) can range from a minimum of -0.117 (a yearly decrease by 1.6 percentage points in the case of Hungary) to a maximum of 0.063 (a yearly increase of 0.9 percentage points in the case of Portugal), with 25\% and 75\% slope quantiles corresponding to -0.025 (-0.34\%) and 0.007 (0.10\%), respectively.
Furthermore, the correlation between slope and intercept is included:
A positive correlation indicates that countries with higher constants (i.e. predicted value in 1995) achieve higher growth over the observation period than countries with lower constants (thereby "extending their lead"), whereas negative correlations indicate that countries with lower constants experience higher growth (thereby "catching up").


main findings:
\begin{itemize}
\item to the extent to which histograms of same variable are different between combinations, countries are not missing at random -> less biased estimates
\item TI decrease
\item smorc increase
\item inequalities show much variation, depending on dataset
\item hwni increase, but very unequally, many 0s
not artifact of construction: many 0 slopes also for low thresholds where there are almost no zeroes
\end{itemize}
\end{document}